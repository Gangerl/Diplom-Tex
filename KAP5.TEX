\chapter{Anwendung auf die medizinischen Daten}
Zur Verf"ugung steht uns die medizinische Datendatei mit 116 Eintr"agen
und je 17 Merkmalen. Diese Merkmale sind: lfd. Nummer, Geschlecht, Alter,
Demenz, Hypertonie, Nikotin, Diabetes, Alkohol, Herzgewicht, Hirngewicht,
Hypercholesterin?, Cholesterinwert, Hypertriglycerid?, Triglyceridwert,
Gewicht, L"ange, SektionsNR und Gehirn-Sektion?
Zur Partitionierung werden haupts"achlich die Verfahren Combined
Algorithms und Simulated Annealing / Localopt verwendet. Als
Zielfunktion dient, wenn nicht anders angegeben, $\sum\sum$.

\begin{figure}[htbp]
%\[ \epsfxsize=8cm \epsfbox{dis000.ps} \]
\caption{Die medizinischen Daten}
\end{figure}

\section{Vorbemerkungen}
Haupts"achlich geht es darum, drei Krankheitstypen der senilen Demenz 
(Form des Schwachsinns) mit den Bezeichnungen MID, SD und SDAT zu erkennen.
Da diese Zuordnung der Datens"atze zu den Auspr"agungen der Demenz als
Merkmal in der Datei vorhanden ist, kann eine jede Partitionierung
leicht auf ihren Erfolg hin "uberpr"uft werden.

Wir werden versuchen eine Partitionierung derart zu finden, da"s sie
mit dem Auftreten der drei Krankheiten weitgehend "ubereinstimmt.
Dementsprechend suchen wir eine Einteilung in drei Cluster.

Da wir nach den Ergebnissen aus Kapitel \ref{tests} nur relativ gleichgro"se
Cluster finden K"onnen, kann es angebracht sein, in mehr als 3 Cluster
zu partitionieren, da die drei Krankheitstypen nicht in gleicher
Anzahl vertreten sind.


\section{Einfache Abh"angigkeiten}
Wird eine einzelne Merkmalsauspr"agung wie Herzgewicht, Hirngewicht, L"ange,
Gewicht, Cholesterinwert oder Triglyceridwert in drei Cluster unterteilt,
lassen sich daraus keine R"uckschl"usse auf die Krankheitsverteilung
ziehen. Dies ergibt eine Kontrolle der Partitionierung, wenn man sie in der
Darstellung "`x--Achse: lfd. Nummer, y--Achse: Demenz"' anschaut.

\section{Komplexere Abh"angigkeiten}
Eine Partitionierung der Menge nach den Merkmalen Hirngewicht, 
Cholesterinwert, Nikotin und Hypertonie brachte auch nicht die
gew"unschte "Ubereinstimmung mit den Krankheiten, wie es in der 
medizinischen Arbeit zu den Daten beschrieben ist.

\section{Weitere Abh"angigkeiten}
Aus der medizinischen Arbeit geht weiter hervor, da"s eine Abh"angigkeit
zwischen dem Herzgewicht und einer bestehenden Hypertonie existiert.
Demzufolge hat man die Auspr"agungen des Herzgewichtes in zwei Cluster
zu partitionieren und dann zu schauen, ob dies mit dem Auftreten der
Hypertonie "ubereinstimmt. Dies war nicht der Fall.

Lediglich nahe liegende Zusammengeh"origkeiten, wie die Abh"angigkeit
von K"or\-per\-l"an\-ge zum Gewicht und die geringere Hirnmasse bei Frauen
konnten festgestellt werden.
