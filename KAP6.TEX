\chapter{Zusammenfassung}
Der Schlu"s der Arbeit besteht aus zwei Teilen: Der erste
beschreibt die Erkentnisse, die "uber die Lokalen Sucheverfahren
gewonnen wurden, der zweite besch"aftigt sich mit den Erfolgen dieser
Verfahren auf den medizinischen Daten.

\section{Lokale Suchverfahren}

Die klassischen lokalen Suchverfahren haben den Nachteil, da"s sie um lokale
Optima herum mehr oder weniger stark oszillieren und somit oft nur 
einen kleinen Bereich des gesamten L"osungsraumes durchsuchen. Daher
ist das Konvergenzverhalten nicht befriedigend.
Abhilfe schaffen hier die kobinierten Verfahren wie Simulated Annealing / 
Localopt und Combined Algorithms, wobei erster besser einen gro"sen 
L"osungsraum durchsuchen kann, und zweiterer sehr schnell L"osungen findet.

Um eine der Anschauung nach optimale Partitionierung auch bei
sichelf"ormigen Gebilden wie in Kapitel \ref{sichel} zu erhalten, 
k"onnte man vielleicht eine neue Zielfunktion entwickeln. Diese m"u"ste 
bei der Bestimmung eines Wertes f"ur einen Cluster nicht die
Distanzen aller Elemente zueinander in diesem Cluster
ber"ucksichtigen, sondern zu jedem Element nur die Distanzen zu den 
unmittelbar benachbarten Elementen.
Damit k"onnte auch das Problem, da"s die Cluster unter den hier getesteten
Zielfunktionen immer in etwa gleich gro"s sind, aus dem Weg geschafft werden.

Lokale Suchverfahren scheinen aber ansonsten sehr vielversprechend zu sein,
um Klassifikationsprobleme zu l"osen.
Daf"ur spricht auch, da"s sie auf Ausrei"ser und Me"sfehler in der
Datenmenge dadurch reagieren, da"s sie lokale Optima finden, bei denen
ein solches Element keinem Cluster vern"unftig zugeordnet werden konnte.

\section{Die medizinischen Daten}
Eine mathematisch optimale Klassifizierung von medizinischen Daten
scheint in der Realit"at wenig sinnvoll zu sein, wie das Kapitel \ref{medizin}
zeigt. Zu viele Eigenschaften, die nicht in Zahlen gepre"st werden 
k"onnen, spielen eine Rolle bei der Entwicklung von Krankheiten.

Nur wenn ein definitives Merkmal eine Krankheit beschreibt, 
k"onnen die getesteten Verfahren dieses ausfindig machen.
Au"serdem sollte f"ur eine fundierte mathematische Betrachtung die
Elementanzahl sehr viel gr"o"ser sein, und die Krankheitstypen
gleichm"a"siger verteilt. Aussagen, die die Menge der Frauen betreffen,
halte ich alle f"ur zu vage, da nur von 25 Frauen die Daten vorliegen.

Bessere Anwendungsgebiete f"ur das Programm \Clustering\ sehe ich
in Bereichen von zoologischen und botanischen Klassifikationssystemen,
bei der Chemikern, die Stoffe nach ihren Eigenschaften klassifizieren
und Elemente im Periodensystem ablegen, bei der Auswahl von Zustellgebieten,
etwa der Verteilung von Postleitzahlen, oder etwa bei einer Untersuchung
des Konsumentenverhaltens von Bev"olkerungsschichten.
Eine andere M"oglichkeit w"are auch noch eine Einteilung in 
Konfektionsgr"o"sen bei Menschen, die nach me"sbaren Daten wie K"orperl"ange,
Gewicht, Umfang, Arml"ange etc. vorgenommen werden kann.


