\begin{appendix}

\chapter{Installation von \Clustering}
Legen Sie die beiliegende Diskette in Ihr Laufwerk und starten Sie das
Program \verb-setup.exe-. Folgen Sie den Anweisungen des Programms, um 
die Installation korrekt durchzuf"uhren.
Alternativ k"onnen Sie die Dateien der Diskette auch manuell auf Ihre
Festplatte kopieren oder das Program \verb-diplom.exe- direkt starten.

Im Verzeichnis \verb-data- finden sie die Testdaten aus dem Kapitel 4.


\chapter{Fehlermeldungen von \Clustering}

\begin{sloppy}
\begin{description}
\item[Anzahl der Cluster zu gross: wird nicht ausgef"uhrt!]
	Die Einf"arbe--Funk\-tion kann nicht ausgef"uhrt werden, da die Anzahl
	der verschiedenen Merkmale in $y$--Richtung die Maximalanzahl der
	Cluster (\MAXCLUSTER) "uerschreitet. W"ahlen Sie f"ur diese Funktion
	ein anderes Merkmal, das in vertikaler Richtung angezeigt werden soll.
\item[Durch Subbereichs--Einschr"ankung keine Punkte mehr, 
	die betrachtet]
	{\bf werden muessen}
	Sie haben zuviele Daten durch die Einschr"ankungen im Subbereiche--Men"u
	ausgeschlossen. Es sind keine Datens"atze "ubriggeblieben. Machen Sie
	eine Ihrer Auswahlen r"uckg"angig.
\item[Kein Speicher in \dots]
	\Clustering\ hatte Probleme beim Allokieren von Speicher. Beenden Sie
	andere Windows--Anwendungen, damit gen"ugend Speicher zur Verf"ugung
	steht und versuchen Sie es dann erneut.
\item[Lesefehler: Daten entsprechen nicht der Vorschrift]
	In der Datei befindet sich ein Datensatz, der nicht den Spezifikationen
	der Formatzeile gen"ugt. Bitte kontroliieren Sie Ihre Datei
	dahingehend (steht z. B. ein Komma statt eines Punktes f"ur eine
	Dezimalstelle?).
\item[Lesefehler: Falsche Titelzeile in der Datei]
	Die Anzahl der durch Semikola getrennten Merkmals"uberschriften stimmt
	nicht mit der Anzahl der Merkmale "uberein oder es treten Leerzeichen auf.
\item[Lesefehler: Feld $n$ ent"alt keine signifikanten Daten]
	In der Merkmalsspalte $n$ tritt nur eine Auspr"agung auf. Bitte
	f"ugen Sie eine weitere Auspr"agung hinzu, oder entfernen Sie diese
	Spalte aus der Datei.
\item[Lesefehler:]
	Allgemeiner Lesefehler
\item[Arbeite auf alten Daten weiter]
	Da das Einlesen einer neuen Datei fehlerhaft verlaufen ist, stehen
	weiterhin die bisherigen Daten zur Verf"ugung.
\end{description}
\end{sloppy}

Alle weiteren Fehlermeldungen stammen von Windows und sollten
im dazugeh"origen Handbuch nachgeschlagen werden.

