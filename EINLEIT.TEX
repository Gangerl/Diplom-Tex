\chapter*{Einleitung}
\addcontentsline{toc}{chapter}{Einleitung}

In der Medizin stellt sich immer wieder das Problem,
anhand von bekannten Faktoren eine Krankheitsdiagnose zu erstellen.
Diese Faktoren k"onnen numerische Daten (z.B. Cholesterinwert),
Zustandswerte (z.B. ja oder nein) oder bestimmte Erscheinungformen
(z.B. Krankheit A oder B oder C) sein.

Diese Arbeit soll nun ein Hilfsmittel zur Verf"ugung stellen, mit dessen Hilfe
solche Entscheidungen leichter gef"allt werden k"onnen:
Ein Patient mit einer gewissen Merkmalsstruktur wird einer
Gruppe von Patienten zugeordnet, die "ahnliche Eigenschaften aufweisen.
Wir haben es also mit einem {\em Klassifikationsproblem} zu tun, zu dessen
L"osung Methoden der {\em Lokalen Suche} verwendet werden sollen;
im Gegensatz zu den statistischen Verfahren aus den 70er Jahren, mit
denen sich bereits viele Autoren eingehend besch"aftigt haben.

Das so entwickelte Softwarepaket \Clustering\ soll einer breiten
Anwenderschicht zur Verf"ugung gestellt werden k"onnen, und wurde deswegen
auf einem Personal Computer unter der grafischen Oberfl"ache 
 Windows\footnote{Windows ist ein eingetragenes Zeichen
der Microsoft Corporation} entwickelt.

%Das Programm \Clustering\, das diese Zuordnung vornimmt, erstellt anhand von zu
%w"ahlenden Merkmalen "uber alle Patienten eine Partitionierung in beliebig
%viele Gruppen (Cluster), sagen wir soviele, wie Krankheiten erwartet werden.

%Da das Programm vornehmlich von Medizinern genutzt werden soll, 
%und Personal Computer immer gr"o"seren Einzug in Bur"os gehalten haben,
%sollte \Clustering\ auf einem solchen Personal Computer entwickelt
%werden, und der Benutzerfreundlichkeit halber unter der grafischen
%Oberfl"ache Windows\footnote{Windows ist ein eingetragenes Zeichen
%der Microsoft Corporation} lauff"ahig sein.

In dieser Arbeit werden zun"achst in Kapitel 1
 Klassifikationsprobleme allgemein besprochen:
Diese sind dadurch charakterisiert, da"s man eine
gegebene endliche Menge von Elementen derart in Teilmengen zerlegen 
m"ochte, da"s diese Teilmengen unter einem bestimmten 
Kriterium optimal sind. In dieser Arbeit werden wir
uns darauf beschr"anken, da"s die Vereinigung all dieser Mengen
stets die ganze Menge ist, und die Teilmengen nicht leer sind.

Das Optimalit"atskriterium wird durch eine Zielfunktion dargestellt, 
mit deren Hilfe es m"oglich ist,
heuristische Verfahren zu entwickeln, die nicht unbedingt die optimale
L"osung finden aber gute brauchbare.
Hierzu verwendet diese Arbeit das Prinzip der {\em Lokalen Suche}, 
das in Kapitel \ref{lokalesuche} n"aher beschrieben wird.
Grundlegend sind hierf"ur Nachbarschaftsstrukturen, die wir dort
ausf"uhrlich diskutieren werden.
Kriterien f"ur dieses Verfahren sind 
%seine gute Anwendbarkeit auf mehrdimensionale Daten und 
seine Anwendbarkeit auf sehr gro"se Datenmengen im
Gegensatz zu exakten Verfahren, und
vor allem die Flexibilit"at durch seine
Zufallssteuerung, die es erm"oglicht auch unscheinbare optimale 
L"osungen zu finden.

In Kapitel 3 wird dann das Programm \Clustering\ vorgestellt.
Nach einer Erl"auterung der implementierten Datenstrukturen
und Algorithmen folgt eine genaue Programmbeschreibung.
 %der
%beiden Programmversionen: Die erste stellt das Testprogramm der
%Algorithmen und Nachbarschaftsstrukturen dar, die zweite Version
%ist das Softwarepaket f"ur den Anwender, in das die Programmteile
%eingeflossen sind, die sich als am besten herauskristallisiert haben.

Im vierten Kapitel wird das Programm \Clustering\ anhand von k"unstlich
generierten Eingabedaten getestet. Es sollen dabei Kenntnisse "uber
gute Kombinationen von Zielfunktion, Nachbarschaft und Suchverfahren
gewonnen werden.

Diese Verfahren werden dann im f"unften Kapitel auf die medizinischen
Daten angewandt. Dabei wird sich zeigen, welche Informationen mit
\Clustering\ gewonnen werden k"onnen.

Den Schlu"s der Arbeit bilden eine kurze Zusammenfassung und einige 
Bemerkungen zu den medizinischen Daten.

Anwendungen f"ur eine Software, die Klassifikationsprobleme l"ost,
hierzu gibt es nicht nur in der Medizin, sondern auch z.B. 
bei Chemikern, die Stoffe mit "ahnlichen Eigenschaften gruppieren, 
oder auch in der Musik, um Musikst"ucke anhand von Lautst"arke, 
Tonfolgen, Rhythmen oder "ahnlichem in  verschiedene Kategorien einzuordnen.
%Kinderlieder oder  Pop zu ordnen.

%Mit Klassifikationsproblemen haben sich bereits viele Autoren
%eingehend be\-sch"af\-tigt, vor allem in den 70'er Jahren. 
%Es wurden haupts"achlich statistische Verfahren aus dem
%Bereich der Wahrscheinlichkeitstheorie und Statistik entwickelt.
%%Wir wollen nun in dieser Arbeit moderne Methoden der Lokalen
%Suche verwenden.

An dieser Stelle m"ochte ich Herrn Prof. Dr. Brucker
f"ur die interessante Themenstellung und die Betreuung w"ahrend
der Arbeit bedanken.
