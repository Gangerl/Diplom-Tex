\chapter{Entwicklung und Handhabung von \Clustering}
Nach den allgemeinen Darstellungen aus Kapitel 1 und 2 sind wir nun
in der Lage uns konkret mit dem medizinischen Problem zu
besch"aftigen. Im ersten Teil dieses Kapitels werden
Hinweise zur Implementierung, wie die Speicherung der Daten und 
Startl"osungen gegeben. Anschlie"send folgen 
einige spezielle Forderungen an das Programm, die aus der Sicht der
Medizin und der Testbarkeit entstanden sind.
Am Ende folgt dann das Handbuch zum Programm \Clustering.

\section{Eingabedaten und Skalierung}
Als Grundlage f"ur die Berechnung m"ussen die $n$ Datens"atze mit 
jeweils $m$ Werten in einer Datenmatrix $X=(x_{ij}) \  (i=1,\dots,n;
\,j=1,\dots,m)$ erfa"st sein, wobei die $x_{ij} \in \R$ sind.
Um eine gleichm"a"sige Wichtung aller Daten zu erhalten, mu"s eine
Skalierung auf das Einheitsintervall durchgef"uhrt werden, d.h.
$x_{ij} \in [0,1]$.

Die Daten kann man in verschiedene Klassen einteilen:
\begin{description}
\item[klassifikatorische Daten,] wie zum Beispiel Blutgruppe
	(endliche viele M"oglichkeiten) oder Fingerabdruck (unendlich
	viele M"og\-lich\-kei\-ten),
\item[komparative Daten,] die auf Skalen angegeben werden k"onnen 
	wie Schulnoten oder Intelligenzquotient,
\item[metrische Daten,] zum Beispiel Temperatur in $ ^\circ C$.
\end{description}
Diese drei Merkmalstypen reichen uns zur Beschreibung 
der Eingabe medizinischer Daten aus.
Klassifikatorische Daten sind bei uns im allgemeinen
Strings (Text), komparative Daten ganzzahlige Werte und metrische Daten
werden durch Gleitkommazahlen dargestellt.

Das Programm \Clustering\ zeigt bei der Visualisierung der Daten und des
Rechenprozesses jeweils das gesamte Intervall von 0 bis 1 auf dem
Bildschirm an. Im Hinblick auf eine gute Darstellbarkeit der Daten
werden die Werte 0 und 1 bei der Skalierung ausgeklammert, 
damit keine Daten auf dem Rand des Fensters liegen.

F"ur Integerzahlen, die eingelesen werden, bedeutet dies, da"s bei einer
Normierung der Daten nicht einfach der Quotient
\[ \frac{\mbox{Merkmalswert} - \mbox{kleinster Merkmalswert}}
	    {\mbox{gr"o"sten} - \mbox{kleinsten Wert dieses Merkmals}} \]
benutzt werden kann, sondern es m"ussen erst ein Maximum und Minimum
definiert werden:
\begin{eqnarray*}
	\mbox{Maximum} & = & \mbox{gr"o"ster Wert} + 1, \\
	\mbox{Minimum} & = & \mbox{kleinster Wert} - 1.
\end{eqnarray*}
Nun erh"alt man f"ur den skalierten Wert $x$ einer Eingabe $e$ den 
Ausdruck
\[ x \ = \ \frac{\mbox{ Integerwert von $e$} - \mbox{Minimum}}
		        {\mbox{Maximum} - \mbox{Minimum dieses Merkmals}}\ . \]

Untersuchungen von Milligan \& Cooper [1988] haben ergeben, da"s eine
solche Normierungsweise am erfolgsversprechendsten ist.
Gleitkommazahlen werden in "ahnlicher Weise behandelt: 
F"ur unsere Zwecke ist es ausreichend, die Zahl nur bis zur 3.
Nachkommastelle zu betrachten, dies dann mit dem Faktor 1000 zu
multiplizieren und als ganze Zahl zu interpretieren.
Dann kann obiges Verfahren angewandt werden.
%Da wir uns 
%nicht mit Daten besch"aftigen wollen, bei denen es auf die vierte 
%Nachkommastelle ankommt, multiplizieren wir den Gleitkommawert mit
%dem Faktor 1000 und interpretieren diese Zahl als ganze Zahl. 
%Dann k"onnen wir obiges Verfahren wieder anwenden.

Bei nichtnumerischen Werten gestaltet sich der Unmrechnungsproze"s 
etwas schwieriger:
Wir ordnen jeder Merkmalsauspr"agung einen ganzzahligen Wert
folgenderma"sen zu:
Jeder neuen Merkmalsauspr"agung eines Merkmals wird die derzeitige
Anzahl der Auspr"agungen $+ 1$ zugeordnet. Das hei"st, da"s der
kleinste zugewiesene Wert die 1, der gr"o"ste die Anzahl der 
Auspr"agungen ist. Auf diese Darstellung 
%k"onnen wir obige Skalierung anwenden.
kann wiederum die Skalierung angewandt werden. Hierbei allerdings
das Problem auf, da"s bei mehr als zwei Auspr"agungen diese 
keine gleichen Absta"ande mehr zueinander aufweisen. Da es sich aber i.a.
um Auspr"agungen der Form "`ja"' und "`nein"' handelt, vernachl"assigen
wir dies.


Ein kleines Beispiel:\\
Die Integerzahlen 1, 3 und 4 werden auf 0.2, 0.6 und 0.8,\\
die Integerzahlen 2 und 4 auf 0.25 und 0.75,\\
die Gleitkommazahlen 0.1, 0.2 und 0.3 auf 1/202, 102/202 und 201/202,\\
die Werte "`ja"' und "`nein"' auf 1/3 und 2/3 abgebildet.

Einzige Einschr"ankung, die diese Verfahren mit sich bringt, ist, 
da"s alle Zahlenwerte $> 0$ sein m"ussen.
Bei weit streuenden Zahlenwerten kann es angebracht sein erst eine
Logarithmierung der Daten vorzunehmen, damit diese besser
gleichverteilt sind. Denn sonst kann eine Verf"alschung der Distanz
zweier Objekte zueinander durch die Skalierung eintreten.

\section{Datenstrukturen}
Die gesamt Speicherverwaltung erfolgt dynamisch, das hei"st es k"onnen
Datenfelder von beinahe beliebiger Gr"o"se eingelesen und verarbeitet werden.
Da das Programm unter Windows 3.1 entwickelt wurde und das Betriebssystem
virtuellen Speicher zur Verf"ugung stellt, entfallen sowohl die sonst f"ur
Personal--Computer unter DOS oft schmerzhafte 640KB--Grenze, als
auch die Hauptspeichergr"o"se.
Die Speicherallokierung basiert weiterhin auf 32--Bit--Zeigern, so da"s
problemlos Speicherbl"ocke gr"o"ser als 64KB verwendet werden k"onnen.
Dies ist auch im Hinblick sp"aterer Portierungen auf 32--Bit Systeme
(Windows 95, Windows NT, OS/2) geschehen.
Die Einhaltung der Regeln f"ur die verschiedenen Speichermodelle des
Windows--Systems erwies sich dabei als "au"serst t"uckisch!

Die Darstellung der Cluster im Rechner erfolgt folgenderma"sen:
Die globalen Integer--Variablen {\tt anzahl, werteanz} und {\tt clusteranz}
speichern die Anzahl der Datens"atze, die Anzahl der Merkmale und die
gew"ahlte Clusteranzahl. Eine Partitionierung wird nun durch den Typen
{\tt Loesung} beschrieben, der wie folgt aufgebaut ist:

\begin{verbatim}
TYPE Loesung = RECORD
    cluster        : ARRAY [1..clusteranz][1..anzahl] OF INTEGER;
    anz_in_cluster : ARRAY [1..clusteranz] OF INTEGER;
    in_cluster     : ARRAY [1..anzahl] OF INTEGER;
    zfktwert       : REAL;
END;
\end{verbatim}
Das 2D--Feld {\tt cluster} enth"alt in der $i$--ten Zeile die Elemente,
die im Cluster $i$ liegen und hat somit {\tt clusteranz}--viele Zeilen.
{\tt anz\_in\_cluster} enth"alt die Anzahl der Elemente in jedem Cluster, 
damit Operationen wie "`Einf"ugen von Elementen in den Cluster"' effizient
ausgef"uhrt werden k"onnen.
Das Feld {\tt in\_cluster} enth"alt die umgekehrte Information: 
In Speicherstelle $i$ steht in welchem Cluster sich das Element $i$ gerade
befindet.
Durch diese zweifache Darstellung f"allt zwar ein gewisser "`Overhead"' bei
der  Speicherverwaltung an, der aber durch einen effizienteren Zugriff mehr als
wett gemacht wird. 
Die Gleitkommazahl {\tt zfktwert} enth"alt den aktuellen 
Zielfunktionswert dieser Partitionierung.

Die Speicherung der Daten in Listen, 
an Stelle von gen"ugend gro"sen Arrays, 
w"are zwar beim Autor sehr beliebt gewesen, 
doch fordert Windows einen zu gro"sen Aufwand
bei dynamischer Speicherverwaltung.

Die eingelesenen, bzw. modifiziert eingelesenen Integerwerte werden in 
einem Feld {\tt input} der Gr"o"se 
{\tt anzahl$\times$clusteranz} gespeichert;
die dazugeh"origen Strings von Texteingaben finden sich in einer
sogenannten String--Tabelle, auf die man mit den entsprechenden
Integerwerten zugreift. Weiter gibt es ein ebenso gro"ses Feld mit
Gleitkommazahlen, das die normierten Werte enth"alt.
Dazu kommt dann noch die symmetrische Distanzmatrix, die sich aufgrund
ihrer Symmetrie besser in einer Dreiecksmatrix, also einem linearen
Array mit Makrozugriff speichern l"a"st.

\section{Startl"osungen}
Auf einen ausgefeilten Startl"osungsmechanismus wurde verzichtet, um nicht
durch eine vorgegebene Ausgangslage den Algorithmus negativ zu beeinflussen.
Vielmehr sollten dem Algorithmus sehr unterschiedliche Startsituationen
zug"anglich gemacht werden, damit er in mehreren Durchl"aufen bessere
Chancen hat das Optimum zu finden. Somit hat sich als echte unabh"angige
Startl"osung nur das Verfahren Random erwiesen:

\begin{algorithm*}
PROCEDURE Startl"osung\_Random (VAR start:Loesung);\\
BEGIN
\begin{Block}
	Initialisiere anz\_in\_cluster mit Nullen;\\
	FOR i := 1 TO anzahl DO BEGIN
	\begin{Block}
		cluster := Rand(1,clusteranz);\\
		start.in\_cluster[i] := cluster;\\
		start.cluster[cluster][anz\_in\_cluster[cluster]] := i;\\
		INC(start.anz\_in\_cluster[cluster]);
	\end{Block}
	END;\\
	Berechne Zielfunktionswert von start;
\end{Block}
END;
\end{algorithm*}

Rand(a,b) liefere dabei eine ganzzahlige Zufallszahl zwischen a und b
einschlie"slich.

\begin{figure}[htp]
\[
\psfig{figure=b3-6.ps}
\]
\caption{Nat"urliche Verteilung}
\end{figure}



\section{Spezielle Programmanforderungen}
Da wir uns speziell mit Daten aus der Medizin besch"aftigen wollen,
brauchen wir noch einige besondere Programmeigenschaften. Dies sind
{\em beliebige Dimensionalit"at}, {\em Subklassenauswahl} und {\em
Zielfunktionenwichtung}.

Beliebige Dimensionalit"at hei"st, da"s der Benutzer unter den Werten
w"ahlen k"onnen soll, welche davon zur Abstandsbestimmung
herangezogen werden. So brauchen die Datens"atze nicht
ver"andert zu werden, wenn ein bestimmtes Merkmal nicht mehr ber"ucksichtigt
werden soll. Entsprechend kann der Benutzer auch die Werte, die auf dem 
Schirm dargestellt werden, frei w"ahlen. Dies ist zum Beispiel dann
n"utzlich, wenn er eine Partitionierung nach 2 Merkmalen vornimmt, sich
davon "uberzeugt, da"s diese Clusterung in Ordnung ist, und dann
auf eine andere Ansicht umschaltet, um Zusammenh"ange zu anderen 
Merkmalen feststellen zu k"onnen.

Die Subklassenauswahl ist eine noch feinere Differenzierung, welche
Punkte und Werte zur Berechnung herangezogen werden sollen.
Hierbei k"onnen Datens"atze g"anzlich unber"ucksichtigt bleiben, wenn
sie bei einem bestimmten Merkmal einen ganz bestimmten Wert aufweisen.
Hat man z.B. das Gef"uhl, das alle Raucher eine vern"unftige
Partitionierung verhindern, da sie z.B.	Ausrei"ser produzieren, so
entscheidet man nur Nichtraucher zu betrachten. 
Dies ist auch sehr n"utzlich, um festzustellen, ob bekannte
Zusammenh"ange in beiden Gruppen (Raucher und Nichtraucher) auftreten,
oder ob sie eine spezielle Charakteristik einer dieser Gruppen sind.

Unter Zielfunktionenwichtung versteht man, da"s die Zielfunktionen auch
kombiniert verwendet werden k"onnen. Gibt es z.B. drei Funktionen, 
die eine ganz bestimmte Menge auf drei verschiedene Weisen ganz gut
partitionieren, so hat man das Bed"urfnis, das Ergebnis herauszufinden, 
wenn man alle drei Funktionen gleichzeitig verwendet.
Das Programm soll also eine M"oglichkeit zur Verf"ugung stellen, die
Zielfunktionen beliebig zu kombinieren und gegeneinander wichten 
zu k"onnen.

Eine kleine, bei langen Berechnungen aber sehr n"utzliche,
Eigenschaft sollte sein:
Au"ser den eingebauten Abbruchkriterien der Algorithmen soll der Benutzer
jederzeit die M"oglichkeit haben durch Dr"ucken der linken Maustaste den
Rechenvorgang zu beenden.

\section{Die Inputdatei}
Um die Inputdatei, die die Daten enth"alt, nicht zu aufwendig einlesen
zu m"ussen werden an sie folgende Forderungen gestellt:

\begin{itemize}
\item ASCII--Datei,
\item Kommentarzeilen beginnen mit \#,
\item 1. Zeile enth"alt die Anzahl der Datens"atze,
\item 2. Zeile beschreibt das Format der Merkmale (s.u.),
\item 3. Zeile enth"alt die Merkmal"uberschriften durch 
	Semikola voneinander getrennt (Titelzeile),
\item alles weiteren Zeilen entsprechen den  Datens"atze, wobei 
	die Merkmale jeweils durch Semikola getrennt sein m"ussen,
\item es d"urfen keine Leerzeichen oder Tabs (Whitespaces)
	auftreten.
\end{itemize}

\subsection*{Die Formatzeile}
Diese Zeile beschreibt den Aufbau der Datens"atze, d.h. sie legt fest,
wie die Werte der einzelnen Spalten zu interpretieren sind.
Eine $0$ steht f"ur ganze Zahlen (Integer),
eine $1$ steht f"ur Gleitkommazahlen (Real), wobei die Dezimalstelle durch 
	einen Punkt "`."' darzustellen ist,
eine $2$ steht f"ur Text (Strings), der keine Leerzeichen enthalten darf. 
(Man schreibe stattdessen z.B. einen Unterstrich "`\_"'.)

Es d"urfen  auch mehr als die in der Formatzeile angegebenen Merkmale
in einem Datensatz auftauchen; diese bleiben dann aber unber"ucksichtigt.

Als sinnvoll hat sich erwiesen, den Zeilen an erster Stelle eine laufende
Nummer zu geben, um die Datens"atze sp"ater besser wiederfinden zu
k"onnen.

\subsection*{Beispiel einer Eingabe}

\begin{verbatim}
#Datei_vom_5.11.__-__Krankheiten
#
10
020212
Nummer;Geschlecht;Alter;Demenz;Zahl;erkrankt?
1;m;86;MID;5.4;ja
2;f;86;MID;4.8;nein
3;m;76;MID;5.1;ja
4;m;79;SD;4.2;nein
5;f;86;SD;4.8;nein
6;m;87;SD;4.5;ja
7;m;89;SD;3.9;nein
8;f;83;MID;5.5;ja
9;m;83;MID;5.5;nein
10;m;70;SD;5.0;nein
\end{verbatim}

\section{Handbuch zu  \Clustering}
\label{handbuch}
Durch dieses Handbuch soll dem Benutzer der Umgang mit \Clustering\
erleichert werden. Er soll ohne weitere Vorkentnisse in der Lage sein
mit \Clustering\ arbeiten zu k"onnen. Die grafische Oberfl"ache soll 
unter Verwendung der "`Online--Hilfe"' weitgehend selbsterkl"arend sein,
da sie auch den Standard Windows--Applikationen weitesgehend 
entspricht. Ein kleines Handbuch sollte jedoch trotzdem nicht fehlen.

Die Realisierung des Programms erfolgte mit Hilfe von {\sc Borland} C++ 4.0
for Windows.
W"ahrend der Entwicklung zeigt sich, da"s dies keine gute Wahl war, denn
es traten einige Compiler--Schw"achen und --Fehler auf, die die
Entwicklungszeit nicht unerheblich verl"angert haben.

Nach dem Programmstart meldet sich
\Clustering\ mit dem Fenster aus Abbildung \ref{startfenster}.

\begin{figure}[ht]
	\[ \epsfxsize=10cm \epsfbox{startfen.ps} \]
	\caption{\Clustering\ nach dem Programmaufruf 
	\label{startfenster}}
\end{figure}

\subsection*{Das {\tt Datei}--Men"u}
Es folgt eine Beschreibung der Men"upunkte des Men"us {\tt Datei}:
\begin{itemize}
\item "Uber {\tt "offnen} (oder Taste {\tt F3})  kann eine Datei
	geladen werden. Diese mu"s obigem Format entsprechen.
	 Sollte dies nicht m"oglich sein, entweder weil die Angaben
	der Formatzeile falsch sind oder g"anzlich fehlen, wird eine
	entsprechende Fehlermeldung ausgegeben.
	Es ist weiter nicht m"oglich, Dateien einzulesen, bei denen eine
	Merkmalsspalte stets den gleichen Eintrag enth"alt (z.B.
	Geschlecht: m"annlich). Dies scheint auch zur Auswertung der Daten
	nicht sinnvoll zu sein, da es sich nicht um ein pr"agendes 
	Merkmal handeln kann.
\item {\tt Clipboard:} Der derzeitige Fensterinhalt wird in das
	Windows--Clipboard kopiert, um von dort aus z.B. mit einem Grafikprogramm
	aufbereitet zu werden.
\item {\tt Drucken:} Die zur Zeit angezeigte Grafik und der dazugeh"orige Text 
	werden auf dem Windows--Standard--Drucker mit Hilfe des 
	Druck--Managers ausgegeben.
\item {\tt Beenden:} Das Programm verlassen.
\end{itemize}

\subsection*{Das {\tt Ansicht}--Men"u}
Der zweite Men"upunkt {\tt Ansicht} des Hauptfensters bietet einige 
Wahlm"oglichkeiten bei der Darstellung der Daten. Dies sind
\begin{itemize}
\item {\tt Cluster-/ Werteausgabe:} Es erscheint ein Fenster mit der 
	Ausgabe der berechneten Cluster, bei
	Clusterausgabe nach Clustern, bei Werteausgabe nach laufenden
	Nummern sortiert. Es werden die lfd (=laufende Nummer)
	der Cluster und
	die an der Berechnung beteiligten Werte anzeigt (siehe Abbildung
	\ref{Abbildungwerteaus}).
	Dies ist nat"urlich erst m"oglich, nachdem eine Berechnung
	stattgefunden hat! 

	\begin{figure}[htp]
	\[ \epsfxsize=12cm \epsfbox{werteaus.ps} \]
		\caption{\label{Abbildungwerteaus}
			Werteausgabe nach laufenden Nummern, 
			in der letzten Spalte stehen die Cluster. 
			Der Button {\tt umschalten} dient dazu, in die 
			andere Darstellung, in diesem Fall die
			Ausgabe nach Clustern, zu wechseln.}
	\end{figure}
\item {\tt mit K.-Kreuz:} Einschalten eines Koordinatenkreuzes (n"utzlich
	zur Orientierung bei 3D--Darstellungen).
\end{itemize}

\subsection*{Der Men"upunkt {\tt Rechnen }}
Durch Auswahl von {\tt Berechnen} 
erfolgt eine Berechnung der neuen Clusterung unter Ber"ucksichtigung 
der unter Optionen und Werte eingestellten
Parameter. Dies kann auch durch Bet"atigen der rechten Maustaste
oder durch Dr"ucken der Taste \verb-F4- erreicht werden.
Um eine Berechnung abzubrechen kann 
die linke Maustaste gedr"uckt werden, und
das Verfahren wird gestoppt.

Mit der Auswahl von {\tt Enumeration} wird eine Routine zur Bestimmung des
optimalen Funktionswertes mittels totaler Enumeration gestartet.
Dies kann allerdings nur bei Problemen mit weniger als 30 Elementen
geschehen, da die Ausf"uhrung sonst zu lange dauern w"urde.
Diese Enumeration ist rekursiv implementiert und betrachtet von 
$k^n$ m"oglichen Partitionen $k^{n-1}$--viele, was nach kurzen
Symmetrie"uberlegungen eine obere Schranke ist. Mittels dynamischer 
Programmierung kann diese Schranke noch ein wenig weiter gesenkt werden,
allerdings k"onnen 
damit auch nur unwesentlich gr"o"sere Probleme gel"ost werden (siehe
hierzu Jensen [1969] und Duran \& Odell [1974]).


\subsection*{Die Dialogbox {\tt Werte }}
Unter {\tt Werte.. }
erscheint ein Dialog, "uber den die anzuzeigenden bzw. zu berechnenden
Werte eingestellt werden k"onnen (siehe Abbildung \ref{Abbildungwertewahl}).
Die Anzahl der Werte f"ur \verb-Werte fuer Anzeige- ist 
dabei auf 3 beschr"ankt (jeder
ben"otigt eine Dimension), w"ahrend zur Berechnung beliebig viele
Werte ausgew"ahlt werden k"onnen: Die weitere Auswahl
unter \verb-weitere Werte- erfolgt durch Klicken  mit der Maus
bei gleichzeitigem Dr"ucken der \verb-SHIFT--- bzw. \verb-Umschalt---Taste.

\begin{figure}[htp]
	\[ \epsfxsize=12cm \epsfbox{wertemen.ps} \]
	\caption{\label{Abbildungwertewahl} Werte--Men"u}
\end{figure}

Mit dem \verb-OK---Button werden diese "ubernommen und das Men"u
verlassen, mit dem Button 
\verb-Abbruch- verl"a"st man das Men"u, ohne da"s die eingestellten
Werte "ubernommen werden, und unter \verb-Hilfe- erscheint
eine kleine Online--Erl"auterung.

Mit der Checkbox \verb-Werte von oben uebernehmen- k"onnen die eingestellten
Werte  f"ur die Anzeige
direkt in das Berechnungsfeld "ubernommen werden.

Der \verb-Einfaerbe---Button dient dazu, eine Einf"arbung nach den derzeitig
angezeigten Werten vorzunehmen, d.h. die Clusteranzahl wird festgelegt durch 
die Anzahl der in $y$--Richtung (vertikal) auftretenden Werte, 
und alle Datens"atze mit dem gleichen Wert werden dem gleichen Cluster
zugeordnet (siehe Abbildung \ref{special}).
Dies ist allerdings nur dann m"oglich, wenn die so gew"ahlte Clusteranzahl
die h"ochstm"ogliche von \MAXCLUSTER\ nicht "uberschreitet.


\begin{figure}[htp]
	\[ \epsfxsize=8cm \epsfbox{einfaerb.ps} \]
	\caption{\label{special} Einf"arbe--Effekt}
\end{figure}

Der Zweck ist, sich diese Clusterung unter anderen Werten anzuschauen, um 
dadurch eine eventuelle H"aufung als Pr"adikat eines Zusammenhangs ausfindig
zu machen.

Eine solche Clusterung l"a"st sich auch erzielen, indem man im
Werte--Auswahl--Men"u unter \verb-Werte fuer Berechnung- nur einen Wert,
n"amlich den, der gerade vertikal verl"auft, angibt. Als Zielfunktion
(siehe \ref{zielfunktionenkapitel}) sollte dabei m"oglichst \verb-SS- (s.u.)
gew"ahlt werden.


\subsubsection*{Der {\tt Unterbereiche}--Dialog}
Der Button \verb-Unterbereiche-  er"offnet ein Men"u, mit dem es m"oglich ist
solche Datens"atze von der Berechnung auszuschlie"sen, die unter einem 
bestimmten
Merkmal einen bestimmten Wert aufweisen (siehe Abbildung \ref{unterber}).
\begin{figure}[htp]
	\[ \epsfxsize=8cm \epsfbox{unterber.ps} \]
	\caption{\label{unterber} Unterbereichs--Men"u}
\end{figure}

Beispielsweise habe ein Merkmal die Auspr"agungen \verb-Ja- und 
\verb-Nein-. Durch Markieren von \verb-Nein- bleiben alle
die Datens"atze unber"ucksichtigt, die dieses \verb-Nein-
aufweisen.
In der Grafik werden weiterhin alle Punkte angezeigt, jedoch
erscheinen diese "`passiven"' Punkte in wei"s.

Dem Benutzer wird dadurch die M"oglichkeit gegeben, Abh"angigkeiten, 
die nur in einer Untergruppe (z.B. Nichtrauchern) auftreten, 
herauszufinden.

\subsection*{Der {\tt Optionen}--Dialog}
In dieser Dialogbox k"onnen die Berechnungsaspekte ver"andert werden.
Hier die Erl"auterungen zu den einzelnen Punkten:
\begin{figure}[htp]
	\[ \epsfxsize=10cm \epsfbox{optionen.ps} \]
	\caption{Optionen--Men"u}
\end{figure}

\begin{description}
\item[{\tt Suchverfahren}]
	Es kann einer der in Kapitel \ref{algorithmeneins} und Kapitel
	\ref{combined} beschriebenen
	Algorithmen (Local Search, Iterative Improvement, Simulated Annealing,
	Simulated Annealing / Localopt, Threshold Acceptance, Combined
	Algorithms, Tabu Search)
	ausgew"ahlt werden. Nach Auswahl der Tabu Suche k"onnen  die
	Tabu Listenl"ange und der Aspiration--Faktor festgelegt werden.
	Bei dem
	Verfahren Simulated Annealing / Localopt gestaltet sich manchmal der
	Abbruch durch den Benutzer (linke Maustaste) etwas schwierig, so
	da"s es "ofter versucht werden sollte.
\item[{\tt Nachbarschaft}]
	Es stehen die drei in Kapitel \ref{nachbarn} vorgestellten 
	Nachbarschaften (Nearest--Cluster, Centroid, Random) zur Auswahl.
	Bei der Nearest--Cluster Nachbarschaft kann dann die Anzahl $t$ der
	zu betrachtenden n"achsten Nachbarpunkte eingestellt werden. 
	Wird hier der Wert 0 gew"ahlt, so ist damit die gesamte 
	Elementanzahl gemeint.

	Wird die Checkbox {\tt Simultaneous} aktiviert, so stehen als 
	Nachbarschaften nur noch Centroid und Nearest--Cluster zur Verf"ugung,
	dar"uber hinaus wird die Auswahlm"oglichkeit des Suchverfahrens
	auf Lokale Suche, Iterative Improvement und Simulated Annealing /
	Localopt beschr"ankt.
\item[{\tt Clusteranzahl}] Hier kann man die gew"unschte Anzahl der Cluster
	eingeben. Voreinstellung ist 3; Maximalanzahl \MAXCLUSTER.
\item[{\tt Wichtung der Zielfunktionen..}] Durch Anw"ahlen dieses Punktes (oder
	durch Dr"u"cken der mittleren Maustaste, soweit vorhanden) erscheint
	ein Men"u, in dem die Wichtung der Zielfunktionen gegeneinander
	ge"andert werden kann (siehe Abbildung \ref{zielfunk}).
	Die Bedeutung der einzelnen Zielfunktionen (vgl. 
	\ref{zielfunktionenkapitel}):

	\begin{center}
	\begin{tabular}{c|c}
	Zeichen & Bedeutung \\
	\hline
	\verb-S S-  &  $F1 = \sum\sum d_{ij}$\\
	\verb-S S^2- &  $F2 = \sum\sum^2 d_{ij}$\\ 
	\verb-S 1/n*S-  & $ F3 = \sum\frac1{|C|}\sum d_{ij}$\\ 
	\verb-max S-  &  $F4 = \sum\max d_{ij}$\\ 
	\verb-S max-  &  $F5 = \max\sum d_{ij}$\\ 
	\verb-max max-  & $ F6 = \max\max d_{ij}$
	\end{tabular}
	\end{center}

	Die Zielfunktionen k"onnen beliebig kombiniert und gegeneinander
	gewichtet werden (dazu dienen die Zahlen --- eine 0 bedeutet, 
	da"s diese Zielfunktion nicht beachtet wird); den vielf"altigen
	M"oglichkeiten des Benutzers sind dabei kaum Grenzen gesetzt.

	Die Voreinstellung ist $1 \cdot$ \verb-SS- und hat sich als recht brauchbar
	erwiesen.

	Wird die Einstellung nach einer Berechnung ver"andert, so wird der 
	neue Zielfunktionswert dieser Partitionierung errechnet und sofort
	angezeigt. Dies erm"oglicht einen guten Vergleich der Ergebnisse,
	wenn unter verschiedenen Zielfunktionen gerechnet wird.

	\begin{figure}[htp]
	\[ \epsfxsize=10cm \epsfbox{zielfunk.ps} \]
		\caption{\label{zielfunk} Zielfunktionen-Men"u}
	\end{figure}

\end{description}

Abbildung \ref{nachbarn} zeigt \Clustering\ nach einer Berechnung.
Dargestellt werden hierbei die angezeigten Werte in $x$-- und $y$--Richtung,
der Zielfunktionswert der Partitionierung, die Anzahl der Iterationen, die
der Algorithmus duchlaufen hat, und die Zeit, die zur Berechnung
ben"otigt wurde. W"ahrend einer Berechnung kann man die Entwicklung der
Zielfunktion durch die Anzeige st"andig verfolgen, wie auch
die dazugeh"orige aktuelle Clusterung st"andig grafisch dargestellt wird.
Bei den Verfahren SA, SA / Localopt, TA und TS kommt noch eine Anzeige
hinzu, die den bislang besten Zielfunktionswert und seine 
Auffindungsdauer angibt.

\subsection*{Hilfe}
Es erscheint eine kontextorientierte Windows--Hilfe im Stile des
Windows--Help--Tools.

Im Anhang A ist kurz beschrieben, wie man unter Hilfe des Setup--Programms
\Clustering\ installiert. Der Anhang B enth"alt eine kurze "Ubersicht
mit Erl"auterungen zu den Fehlermeldungen. Anhang C zeigt eine
m"ogliche Vorgehensweise, um Strukturen in der medizinischen Datenbank
zu erkennen.
