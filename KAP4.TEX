\chapter{Testreihen}
\label{tests}
Um die Verfahren testen zu k"onnen, werden entsprechende Testdaten
ben"otigt. In diesem Kapitel werden dazu
einige k"unstlich erzeugte Eingabedateien betrachtet,
die jeweils spezielle Charakteristika aufweisen. Mit diesen Daten sollen
R"uckschl"usse auf sinnvolle Kombinationen von Nachbarschaften, Algorithmen 
und Zielfunktionen gewonnen werden. 

Alle Laufzeiten beziehen sich auf einen PC mit 486DX2--Prozessor (66MHz)
und 20MB RAM unter Windows 3.11

\section{Betrachtung einer gleichverteilten Menge}
Als erstes wollen wir ein Problem betrachten, bei dem die Elemente 
weitestgehend gleichverteilt sind. Siehe dazu die Abbildung 
\ref{gleichverteilt}. Die Anzahl der Elemente betr"agt 30.

\begin{figure}[htbp]
%\[ \epsfxsize=4cm \epsfbox{gleichve.ps} \]
\caption{Gleichverteilte Elemente}
\label{gleichverteilt}
\end{figure}

Es wurden jeweils 5 Testl"aufe zu jeder betrachteten Kombination
durchgef"uhrt. Angegeben sind dabei der Zielfunktionswert und die Zeit,
die ben"otigt wurde, diesen zu finden.  
Unter "`Sv."' ist das Suchverfahren angegeben. Es bedeuten dabei
LS = Local Search, II = Iterative Improvement, SA = Simulated Annealing,
SL = Simulated Annealing / Localopt, TA = Threshold Acceptance, 
TS = Tabu Search.
Die Tests sind in die Nachbarschaftsauswahl untergliedert.


\subsection*{Single--Nearest--Cluster}
In dieser Teiluntersuchung wollen wir auch einen optimalen Wert
f"ur die Anzahl der Nachbarn bei der Nearest--Cluster Nachbarschaft 
bestimmen. Ein optimaler Wert ist sicherlich die Anzahl aller Elemente,
aber wie wir uns "uberlegt haben, ist dieser Wert i.a.\ viel zu gro"s.
Der erste Teil einer Tabelle beschreibt die Werte, die erzielt wurden, 
wenn alle Nachbarn betrachtet werden.

Ergebnisse f"ur $k=2$ Cluster, optimaler Wert 94.80 f"ur die Zielfunktion 
$\sum\sum$:

{\footnotesize
\setlength\tabcolsep{0.75mm}
\begin{tabular}{l||rr|rr|rr|rr|rr||ccc||ccc}
%\multicolumn{1}{c}{} 
Sv. & 
	\multicolumn{10}{c||}{{\normalsize Ergebniswerte} \rule[-1mm]{0mm}{5mm} } &
	\multicolumn{3}{c||}{{\normalsize Zeit}} & 
	\multicolumn{3}{c}{{\normalsize Zfkt.-wert}} \\ 
%\raisebox{2.5ex}[-2.5ex]{{\normalsize Sv.}} &
 &
	\multicolumn{2}{c|}{1} &
	\multicolumn{2}{c|}{2} &
	\multicolumn{2}{c|}{3} &
	\multicolumn{2}{c|}{4} &
	\multicolumn{2}{c||}{5} &
	min & max & {\o} & min & max & {\o}\\
\hline
\multicolumn{17}{c}{{\normalsize Nachbaranzahl $t=30$:}\rule[-2mm]{0mm}{8mm}}\\
\hline
LS&95.80&0.49&95.80&0.28&94.80&0.38&94.80&0.44&94.80&0.28&0.28&0.49&0.37&94.80&95.80&95.20\\
II&95.74&0.28&95.74&0.22&95.80&0.22&95.80&0.22&95.80&0.22&0.22&0.28&0.23&95.74&95.80&95.78\\
SA&94.80&0.60&94.80&0.61&94.80&0.55&94.80&0.66&97.76&0.66&0.55&0.66&0.62&94.80&97.76&95.39\\
SL&94.80&5.88&94.80&3.80&94.80&4.78&94.80&2.64&94.80&0.66&0.66&5.88&3.55&94.80&94.80&94.80\\
TA&94.80&0.50&95.80&0.71&95.80&0.71&94.80&0.55&95.74&0.77&0.44&0.77&0.65&94.80&95.80&95.39\\
TS&96.10&0.93&96.72&0.82&95.74&0.77&95.80&0.60&95.74&0.49&0.49&0.93&0.72&95.74&96.72&96.02\\
\hline
\multicolumn{1}{l}{{\o}} & \multicolumn{10}{c}{\quad} & 0.49&1.50&\multicolumn{1}{c}{1.02} & 95.11&96.11&95.43\\
\multicolumn{17}{c}{{\normalsize Nachbaranzahl $t=5$:}\rule[-2mm]{0mm}{8mm}}\\
\hline
LS&94.80&0.39&94.80&0.39&94.80&0.39&99.32&0.44&97.76&0.17&0.17&0.44&0.36&94.80&98.32&96.30\\
II&95.80&0.17&94.80&0.11&95.74&0.16&94.80&0.17&95.80&0.17&0.11&0.17&0.16&94.80&95.80&95.39\\
SA&94.80&0.60&94.80&0.60&94.80&0.55&94.80&0.66&94.80&0.61&0.55&0.66&0.60&94.80&94.80&94.80\\
SL&94.80&4.01&94.80&3.24&94.80&2.53&94.80&1.81&94.80&4.23&1.81&4.23&3.16&94.80&94.80&94.80\\
TA&94.80&0.50&97.76&0.60&95.74&0.49&95.74&0.49&95.74&0.60&0.49&0.60&0.54&94.80&97.76&95.96\\
TS&94.80&0.71&94.80&0.22&94.80&1.26&94.80&1.21&94.80&0.88&0.22&1.26&0.86&94.80&94.80&94.80\\
\hline
\multicolumn{1}{l}{{\o}} & \multicolumn{10}{c}{\quad} & 0.56&1.23&\multicolumn{1}{c}{0.95} & 94.80&96.05&95.34\\
\multicolumn{17}{c}{{\normalsize Nachbaranzahl $t=3$:}\rule[-2mm]{0mm}{8mm}}\\
\hline
LS&99.32&0.28&96.69&0.22&96.72&0.17&99.32&0.33&99.41&0.33&0.17&0.33&0.27&96.69&99.41&98.29\\ 
II&99.51&0.06&96.69&0.11&94.80&0.11&96.77&0.11&99.32&0.11&0.06&0.11&0.10&94.80&99.51&97.42\\
SA&94.80&0.55&98.73&0.50&98.73&0.49&95.74&0.49&98.73&0.49&0.49&0.55&0.50&94.80&98.73&97.35\\
SL&96.72&2.25&94.80&1.81&95.80&1.15&94.80&0.38&97.76&2.28&0.38&2.25&1.57&94.80&97.76&95.98\\
TA&95.74&0.38&97.76&0.50&96.77&0.44&96.69&0.28&94.80&0.49&0.28&0.50&0.42&94.80&97.76&96.35\\
TS&95.74&0.99&96.64&1.09&94.80&1.26&94.80&0.17&96.64&1.04&0.17&1.26&0.91&94.80&96.64&95.72\\
\hline
\multicolumn{1}{l}{{\o}} & \multicolumn{10}{c}{\quad} & 0.26&0.83&\multicolumn{1}{c}{0.63} & 95.12&98.30&96.85
\end{tabular}
}

\vspace{3mm}

Die teilweise recht unterschiedlichen Ergebnisse in einer Zeile lassen 
darauf schliessen, da"s die L"osung stark von der Startpartition abh"angt.
Dementsprechend ist es also sinnvoll, f"ur ein konkretes Problem ein
Verfahren mehrfach anzuwenden, um den besten Zielfunktionswert zu bestimmen.
Dieses Verhalten l"a"st sich auch genau bei dem Verfahren Simulated Annealing /
Localopt beobachten, bei dem durch mehrfache Anwendung der Lokalen
Suche fast immer das Optimum gefunden wird. Allerdings mu"s daf"ur auch
eine entsprechend h"ohere Rechenzeit in Kauf genommen werden.

Bei der Tabu Suche hat sich eine
L"ange von 3 Eintr"agen in der Tabu Liste als optimal 
erwiesen. Ist die Liste k"urzer, so konvergiert die Tabu Suche
deutlich schlechter; je l"anger sie ist, desto geringer ist die
Wahrscheinlichkeit, das Optimum zu erreichen.
Interessanterweise sind die Ergebnisse bei einer kleineren Nachbaranzahl
deutlich besser, als wenn alle Elemente bei der Suche nach einem neuen
Cluster betrachtet werden.

Nach unseren "Uberlegungen erwarten wir bei der gr"o"sten Nachbaranzahl die 
besten Zielfunktionswerte; und bei kleineren Anzahlen schlechtere
Werte bei k"urzerer Laufzeit. Dies ist nicht unbedingt
erf"ullt, denn die Zielfunktionswerte sind f"ur die Nachbaranzahl $t=5$ sogar
noch besser (im Durchschnitt 95.34 zu 95.43, au"ser bei der Zielfunktion
Threshold Acceptance). 
%Dies kann nat"urlich an einer zu kleinen Stichprobe der Me"swerte liegen. 
Eine m"ogliche Ursache hierf"ur ist ein zu kleiner Stichprobenumfang.

Wird die Nachbaranzahl weiter
gesenkt, wie im dritten Teil der Tabelle auf $t=3$ Nachbarn, 
so stellt man fest, da"s die
Ergebnisse wieder schlechter werden. Drei Nachbarn scheinen also nicht
auszureichen, um gleichm"a"sig gute Ergebnisse zu erzielen.
Betrachtet man die Laufzeiten, so sieht man 
best"atigt, da"s die Algorithmen bei weniger Nachbarn schneller
terminieren; bei 3 zu 30 Nachbarn beinahe in der H"alfte der Zeit.

Diese Ergebnisse lassen sich f"ur eine gr"o"sere Clusteranzahl ($k=3,4,5$)
in so weit best"atigen, da"s bei 5 Nachbarn noch sehr gute und bei weniger
Nachbarn keine so guten Ergebnisse mehr erzielt werden.
Der einzige Unterschied besteht darin, da"s die Ergebnisse bei 5 Nachbarn nicht
mehr besser sind als bei mehr Nachbarn, was bei obiger Tabelle auf eine
"`ungl"uckliche"' Stichprobe schliessen l"a"st.

Die Tabellen hierzu sind nicht extra aufgef"uhrt, da sie au"ser den eben
genannten keine weiteren Informationen beinhalten. Ebenso verh"alt es sich
mit den Daten f"ur die Zielfunktion $\sum\sum^2$, die die gleiche
L"osungspartition mit dem Zielfunktionswert 51.57 liefert. "Uber die 
anderen Zielfunktionen l"a"st sich leider nur sagen, da"s sie keine
sinnvollen L"osungen lieferten; entweder zeigen sie keine 
Kon\-ver\-genz\-ei\-gen\-schaf\-ten, oder sie sind bestrebt, alle Elemente einem
Cluster zuzuordnen.


\subsection*{Single--Centroid}

Ergebnisse f"ur $k=2$ Cluster, optimaler Wert 94.80 f"ur die Zielfunktion 
$\sum\sum$:

{\footnotesize
\setlength\tabcolsep{0.75mm}
\begin{tabular}{l||rr|rr|rr|rr|rr||ccc||ccc}
%\multicolumn{1}{c}{} & 
Sv. & 
	\multicolumn{10}{c||}{{\normalsize Ergebniswerte} \rule[-1mm]{0mm}{5mm} } &
	\multicolumn{3}{c||}{{\normalsize Zeit}} & 
	\multicolumn{3}{c}{{\normalsize Zfkt.-wert}} \\ 
%\raisebox{2.5ex}[-2.5ex]{{\normalsize Sv.}} &
 &
	\multicolumn{2}{c|}{1} &
	\multicolumn{2}{c|}{2} &
	\multicolumn{2}{c|}{3} &
	\multicolumn{2}{c|}{4} &
	\multicolumn{2}{c||}{5} &
	min & max & {\o} & min & max & {\o}\\
\hline
\hline
LS&94.80&0.39&95.80&0.33&94.80&0.33&99.81&0.27&99.32&0.38&0.27&0.39&0.34&94.80&99.81&96.91\\
II&95.74&0.33&99.35&0.17&96.05&0.22&94.80&0.27&98.73&0.33&0.17&0.33&0.26&94.80&99.35&96.93\\
SA&94.80&0.22&94.80&0.17&95.74&0.22&94.80&0.22&95.74&0.28&0.17&0.28&0.22&94.80&95.74&95.18\\
SL&94.80&0.28&97.76&0.22&94.80&0.33&98.73&0.38&94.80&0.39&0.22&0.39&0.32&94.80&98.73&96.18\\
TA&94.80&0.22&97.90&0.22&94.80&0.22&99.66&0.22&97.76&0.11&0.11&0.22&0.20&94.80&99.88&98.01\\
TS&96.72&0.22&99.32&0.27&99.32&0.28&99.88&0.17&94.80&0.22&0.17&0.28&0.23&94.80&99.88&98.01\\
\hline
\multicolumn{1}{l}{{\o}} & \multicolumn{10}{c}{\quad} & 0.19&0.32&\multicolumn{1}{c}{0.26} & 94.80&98.88&96.70
\end{tabular}
}

Die Algorithmen zeigen alle ein "ahnliches Verhalten wie bei der 
Nearest--Cluster Nachbarschaft, jedoch sind hier die durchschnittlichen
Zielfunktionswerte deutlich schlechter. Andererseits terminieren die 
Verfahren auch schneller. Die Single--Centroid Nachbarschaft ist dar"uber
hinaus in der Lage f"ur alle Zielfunktionen bei entsprechenden Algorithmen,
in erster Linie Simulated Annealing / Localopt, eine L"osung zu finden.
Dies ist in Tabelle \ref{tabgleich} dargestellt.

\subsection*{Single--Random}

Ergebnisse f"ur $k=2$ Cluster, optimaler Wert 94.80 f"ur die Zielfunktion 
$\sum\sum$:

{\footnotesize
\setlength\tabcolsep{0.75mm}
\begin{tabular}{l||rr|rr|rr|rr|rr||ccc||ccc}
%\multicolumn{1}{c}{} & 
Sv. & 
	\multicolumn{10}{c||}{{\normalsize Ergebniswerte} \rule[-1mm]{0mm}{5mm} } &
	\multicolumn{3}{c||}{{\normalsize Zeit}} & 
	\multicolumn{3}{c}{{\normalsize Zfkt.-wert}} \\ 
%\raisebox{2.5ex}[-2.5ex]{{\normalsize Sv.}} &
 & 
	\multicolumn{2}{c|}{1} &
	\multicolumn{2}{c|}{2} &
	\multicolumn{2}{c|}{3} &
	\multicolumn{2}{c|}{4} &
	\multicolumn{2}{c||}{5} &
	min & max & {\o} & min & max & {\o}\\
\hline
\hline
LS&98.73&0.28&98.73&0.28&94.80&0.33&95.74&0.22&95.74&0.39&0.22&0.39&0.30&94.80&98.73&96.75\\
II&95.80&0.17&98.73&0.17&95.74&0.11&99.33&0.11&95.80&0.22&0.11&0.22&0.16&95.80&99.33&97.08\\
SA&95.74&0.71&95.74&0.60&94.80&0.60&94.80&0.93&94.80&0.77&0.60&0.93&0.72&94.80&95.74&95.18\\
SL&95.42&4.78&96.31&2.21&94.80&2.86&94.80&4.01&94.80&0.99&0.99&4.78&2.97&94.80&96.31&95.23\\
TA&94.80&0.88&95.74&0.77&99.32&0.50&95.74&0.60&99.32&0.60&0.50&0.88&0.67&95.74&99.32&96.98\\
TS&95.80&1.09&94.80&1.37&95.74&1.98&97.76&0.71&96.82&1.22&0.51&1.53&1.02&95.12&97.87&96.23\\
\hline
\multicolumn{1}{l}{{\o}} & \multicolumn{10}{c}{\quad} & 0.51&1.53&\multicolumn{1}{c}{1.02} & 95.12&97.87&96.23
\end{tabular}
}

Auch f"ur diese Nachbarschaft werden keine so guten L"osungen gefunden, und
auch die Laufzeiten der Algorithmen sind diesmal deutlich schlechter.
Es zeigt sich, da"s eine zuf"allige Nachbarauswahl nicht die Leistung der
anderen Nachbarschaften bringen kann. Als Zielfunktion kommt auch hier
nur noch $\sum\sum^2$ in Frage.

\subsection*{Simultaneous--Nearest--Cluster}
Es hat sich gezeigt, da"s die Simultaneous--Nearest--Cluster Nachbarschaft 
die Erwartungen nicht erf"ullen konnte: Entweder verlief
der Schritt von einer Partition zur n"achsten zu ungeordnet, so da"s der
neue Zielfunktionswert nicht besser war als der alte; oder
es wurden alle Elemente einem einzigen Cluster zugeordnet, was keine
sinnvolle L"osung ist.
Folglich wird diese Nachbarschaft in die weiteren Betrachtungen
nicht mit einbezogen.

\subsection*{Simultaneous--Centroid}

Ergebnisse f"ur $k=2$ Cluster, optimaler Wert 94.80 f"ur die Zielfunktion 
$\sum\sum$:

{\footnotesize
\setlength\tabcolsep{0.75mm}
\begin{tabular}{l||rr|rr|rr|rr|rr||ccc||ccc}
%\multicolumn{1}{c}{} & 
Sv. & 
	\multicolumn{10}{c||}{{\normalsize Ergebniswerte} \rule[-1mm]{0mm}{5mm} } &
	\multicolumn{3}{c||}{{\normalsize Zeit}} & 
	\multicolumn{3}{c}{{\normalsize Zfkt.-wert}} \\ 
% \raisebox{2.5ex}[-2.5ex]{{\normalsize Sv.}} &
 & 
	\multicolumn{2}{c|}{1} &
	\multicolumn{2}{c|}{2} &
	\multicolumn{2}{c|}{3} &
	\multicolumn{2}{c|}{4} &
	\multicolumn{2}{c||}{5} &
	min & max & {\o} & min & max & {\o}\\
\hline
\hline
LS&95.74&0.05&95.80&0.01&96.22&0.05&94.80&0.16&95.74&0.05&0.01&0.16&0.06&94.80&96.22&95.66\\
SL&95.74&0.11&94.80&0.05&94.80&0.05&95.74&0.05&95.74&0.11&0.05&0.11&0.08&94.80&95.74&95.36\\
\hline
\multicolumn{1}{l}{{\o}} & \multicolumn{10}{c}{\quad} & 0.03&0.14&\multicolumn{1}{c}{0.07} & 94.80&95.98&95.51
\end{tabular}
}

Die wesentlich k"urzeren Laufzeiten im Gegensatz zu den Single Nachbarschaften
beeintr"achtigen nicht die F"ahigkeit dieser Nachbarschaft, gute 
L"osungen zu finden. Dar"uberhinaus werden unter allen Zielfunktionen
L"osungen gefunden.


\subsection*{Combined Algorithms}

Ergebnisse f"ur $k=2$ Cluster, optimaler Wert 94.80 f"ur die Zielfunktion 
$\sum\sum$:

{\footnotesize
\setlength\tabcolsep{0.75mm}
\begin{tabular}{l||rr|rr|rr|rr|rr||ccc||ccc}
%\multicolumn{1}{c}{} & 
Sv. & 
	\multicolumn{10}{c||}{{\normalsize Ergebniswerte} \rule[-1mm]{0mm}{5mm} } &
	\multicolumn{3}{c||}{{\normalsize Zeit}} & 
	\multicolumn{3}{c}{{\normalsize Zfkt.-wert}} \\ 
%\raisebox{2.5ex}[-2.5ex]{{\normalsize Sv.}} &
 & 
	\multicolumn{2}{c|}{1} &
	\multicolumn{2}{c|}{2} &
	\multicolumn{2}{c|}{3} &
	\multicolumn{2}{c|}{4} &
	\multicolumn{2}{c||}{5} &
	min & max & {\o} & min & max & {\o}\\
\hline
\hline
CA&94.80&0.17&94.80&0.11&95.74&0.11&94.80&0.11&94.80&0.17&0.11&0.17&0.13&94.80&95.74&94.99
\end{tabular}
}

Diese Weiterentwicklung von Simultaneous--Centroid zeigt, wie erwartet,
bei ein wenig l"angerer Laufzeit ein besseres Verhalten beim Auf\/finden
einer optimalen L"osung.
Als Zielfunktion kommen aber die beiden Funktionen, die erst das Maximum
innerhalb eines Clusters bestimmen, nicht in Frage.

Folgende Tabelle soll einen "Uberblick 
geben, welche Zielfunktionen mit Nachbarschaften und Algorithmen 
korrespondieren:

\begin{table}[htb]
\caption{Kombinationen von Zielfunktion, Nachbarschaft und Suchverfahren, die 
f"ur das Problem einer gleichverteilten Menge 
\label{tabgleich}
brauchbare Ergebnisse liefern. Es bedeuten hierbei: NC = Nearest Cluster, 
C = Centroid, R = Random, SC = Simultaneous Centroid.}
\begin{center}
\begin{tabular}{l||c|c|c|c|c|c}
   & $\sum\sum$ & $\sum\sum^2$ & $\sum\frac 1n \sum$ & $\max\sum$ & $\sum\max$ & $\max\max$\\
\hline\hline
LS &\scriptsize{\begin{tabular}{c}NC\\C\\R\\SC\end{tabular}}&
	\scriptsize{\begin{tabular}{c}NC\\C\\R\\SC\end{tabular}}&
	\scriptsize{\begin{tabular}{c}\\\\\\SC\end{tabular}}&
	\scriptsize{\begin{tabular}{c}\\\\\\SC\end{tabular}}&
	\scriptsize{\begin{tabular}{c}\\C\\\\SC\end{tabular}}&
	\scriptsize{\begin{tabular}{c}\\C\\\\SC\end{tabular}}
	\\ \hline
II &\scriptsize{\begin{tabular}{c}NC\\C\\R\\SC\end{tabular}}&
	\scriptsize{\begin{tabular}{c}NC\\C\\R\\SC\end{tabular}}&
	\scriptsize{\begin{tabular}{c}\\\\\\SC\end{tabular}}&
	\scriptsize{\begin{tabular}{c}\\\\\\SC\end{tabular}}&
	\scriptsize{\begin{tabular}{c}\\\\\\SC\end{tabular}}&
	\scriptsize{\begin{tabular}{c}\\\\\\SC\end{tabular}}
	\\ \hline
SA &\scriptsize{\begin{tabular}{c}NC\\C\\R\\\quad\end{tabular}}&
	\scriptsize{\begin{tabular}{c}NC\\C\\R\\\quad\end{tabular}}&
	\scriptsize{\begin{tabular}{c}\\\\\\\quad\end{tabular}}&
	\scriptsize{\begin{tabular}{c}\\C\\\\\quad\end{tabular}}&
	\scriptsize{\begin{tabular}{c}\\C\\\\\quad\end{tabular}}&
	\scriptsize{\begin{tabular}{c}\\C\\\\\quad\end{tabular}}
	\\ \hline
SL &\scriptsize{\begin{tabular}{c}NC\\C\\R\\SC\end{tabular}}&
	\scriptsize{\begin{tabular}{c}NC\\C\\R\\SC\end{tabular}}&
	\scriptsize{\begin{tabular}{c}\\C\\\\SC\end{tabular}}&
	\scriptsize{\begin{tabular}{c}\\C\\\\SC\end{tabular}}&
	\scriptsize{\begin{tabular}{c}\\C\\\\SC\end{tabular}}&
	\scriptsize{\begin{tabular}{c}\\C\\\\SC\end{tabular}}
	\\ \hline
TA &\scriptsize{\begin{tabular}{c}NC\\C\\R\\\quad\end{tabular}}&
	\scriptsize{\begin{tabular}{c}NC\\C\\\\\quad\end{tabular}}&
	\scriptsize{\begin{tabular}{c}\\\\\\\quad\end{tabular}}&
	\scriptsize{\begin{tabular}{c}\\C\\\\\quad\end{tabular}}&
	\scriptsize{\begin{tabular}{c}\\C\\\\\quad\end{tabular}}&
	\scriptsize{\begin{tabular}{c}\\C\\\\\quad\end{tabular}}
	\\ \hline
TS &\scriptsize{\begin{tabular}{c}NC\\C\\R\\\quad\end{tabular}}&
	\scriptsize{\begin{tabular}{c}NC\\C\\R\\\quad\end{tabular}}&
	\scriptsize{\begin{tabular}{c}\\\\\quad\\\quad\end{tabular}}&
	\scriptsize{\begin{tabular}{c}\\\\\\\end{tabular}}&
	\scriptsize{\begin{tabular}{c}\\\\\\\end{tabular}}&
	\scriptsize{\begin{tabular}{c}\\\\\\\end{tabular}}
	\\ \hline\hline
CA & ja & ja & ja & ja & &
\end{tabular}
\end{center}
\end{table}


\section{Betrachtung einer Menge, die aus 3 Clustern besteht}
\label{kap42}
Ein weiteres Testbeispiel ist eine Menge, bei der die Elemente klar
in drei Teilmengen unterteilt sind. Die Algorithmen sollten also in der Lage 
sein, diese Struktur schnell zu erkennen. Die Ergebnisse dieser Testreihe 
sollen Aussagen treffen, welche Verfahren "uberhaupt in der Lage sein werden,
Partitionen zu erkennen und mit welcher Geschwindigkeit.
Die Verteilung der Elemente und eine optimale L"osung sind in 
Abbildung \ref{3haufen} dargestellt. Die Anzahl der Elemente betr"agt 40.

\begin{figure}[htbp]
%\[ \epsfxsize=4cm \epsfbox{3haufen.ps} \]
\caption{Menge aus 3 Clustern}
\label{3haufen}
\end{figure}

Durchschnittliche Laufzeiten in Sekunden zur Auf\/findung des Optimums. 
Ergebnisse f"ur $k=3$ Cluster. Ein Strich ("`---"') bedeutet hierbei, da"s
das Optimum nicht gefunden wurde.

\subsection*{Single--Nearest--Cluster}

\begin{center}
\begin{tabular}{l|c|c|c}
 & $\sum\sum,\ \sum\sum^2$ & $\sum \frac 1n \sum,\ \sum\max,\ \max\sum$ &
 	$\max\max$\\
	\hline
LS & 0.55     & --- & 5\\
II & 0.41     & --- & 3\\
SA & 0.61     & --- & ---\\
SL & 0.58     & --- & 5\\
TA & 0.58     & --- & 2.5\\
TS & 0.49$^*$ & --- & 2.5$^*$
\end{tabular}\\
\end{center}
$^*$ Es treten F"alle auf, bei denen Tabu Suche nur sehr
langsam konvergiert (5-12 Sekunden!).\\


Die Anzahl $t$ der Nachbarn war hier mit 3 v"ollig ausreichend, eine
gr"o"sere Zahl veranlasste die drei nicht konvergierenden Algorithmen
nicht zu besseren L"osungen. Da wir bei dieser Problemstellung  mit 40 
Elementen weniger Nachbarn ben"otigen als bei dem ersten Testbeispiel,
k"onnen wir schlie"sen: Die  ben"otigte Nachbaranzahl ist nicht von
der Gesamtanzahl der Elemente, sondern vielmehr von der 
Verteilungsstruktur abh"angig.

\subsection*{Single--Centroid}
\begin{center}
\begin{tabular}{l|c|c|c|c|c}
 & $\sum\sum,\ \sum\sum^2$ & $\sum \frac 1n \sum$ & $\sum\max$ & $ \max\sum$ &
 	$\max\max$\\
	\hline
LS & 0.52 & ---  & ---$^*$ & 0.68    & 0.63 \\
II & 0.97 & ---  & ---     & ---$^*$ & ---$^*$ \\
SA & 0.38 & ---  & 0.40    & 0.35    & 0.38 \\
SL & 0.60 & 3.20 & 1.20    & 0.90    & 0.71 \\
TA & 0.38 & ---  & 0.28    & 0.44    & 0.50 \\
TS & 0.79 & ---  & 0.92    & 1.06    & 0.89 
\end{tabular}\\
\end{center}
$^*$ Es wurde nur in vereinzelten F"allen das Optimum gefunden.

Die Zielfunktionen, unter denen die Algorithmen L"osungen finden, sind
im wesentlichen die gleichen wie beim ersten Beispiel.
Zus"atzlich war hier die Tabu Suche bei den maximierenden Funktionen
erfolgreich.

\subsection*{Single--Random}
\begin{center}
\begin{tabular}{l|c|c|c|c|c}
 & $\sum\sum,\ \sum\sum^2$ & $\sum \frac 1n \sum$ & $\sum\max$ & $ \max\sum$ &
 	$\max\max$\\
	\hline
LS & ---$^*$ & ---     & --- & --- & --- \\
II & ---$^*$ & ---$^*$ & --- & --- & --- \\
SA & 0.82    & ---     & --- & --- & --- \\
SL & 0.80    & ---     & --- & --- & 23.7\\
TA & 0.79    & ---     & --- & --- & --- \\
TS & ---$^*$ & ---     & --- & --- & --- 
\end{tabular}\\
\end{center}
$^*$ Es wurde nur in vereinzelten F"allen das Optimum gefunden.

Die geringe Ausbeute der Kombinationen, die die L"osung finden konnten,
zeigt wieder das schlechte Verhalten dieser Nachbarschaft. Lediglich
die Algorithmen, die Spr"unge "uber lokale Optima zulassen, 
konnten "uberzeugen.

\subsection*{Simultaneous--Nearest--Cluster}
Wie auch schon im ersten Test hat diese Nachbarschaft keine verwertbaren
Ergebnisse geliefert.

\subsection*{Simultaneous--Centroid}
\begin{center}
\begin{tabular}{l|c|c|c|c|c}
 & $\sum\sum,\ \sum\sum^2$ & $\sum \frac 1n \sum$ & $\sum\max$ & $ \max\sum$ &
 	$\max\max$\\
	\hline
LS$^*$ & 0.17 & 0.13 & 0.13 & 0.13 & 0.16 \\
SL     & 0.18 & 0.18 & 0.25 & 0.21 & 0.17 
\end{tabular}\\
\end{center}
$^*$ In vereinzelten F"allen wurde das Optimum {\bf nicht} gefunden.

Die wieder sehr kurzen Laufzeiten best"atigen die Aussage des ersten Teils
"uber diese Nachbarschaft. Da der einfache Simultaneous Algorithmus
nicht immer in der Lage war, die L"osung zu finden, wird best"atigt,
da"s eine Nachoptimierung ben"otigt wird.

\subsection*{Combined Algorithms}
\begin{center}
\begin{tabular}{l|c|c|c|c|c}
 & $\sum\sum,\ \sum\sum^2$ & $\sum \frac 1n \sum$ & $\sum\max$ & $ \max\sum$ &
 	$\max\max$\\
	\hline
CA & 0.22 & 0.21 & 0.42$^*$ & 0.19$^*$ & 1.30
\end{tabular}\\
\end{center}
$^*$ In vereinzelten F"allen wurde das Optimum {\bf nicht} gefunden.

Mit Hife der Nachoptimierung zeigt sich, wie bei Simulated Annealing /
Localopt, ein besseres Konvergenzverhalten beim Finden der L"osung.

%\section{Betrachtung zweier sich umfassender Mengen}
\section{Betrachtung einer eine andere sichelf"ormig umfassender Menge}
\label{sichel}
Dieses Beispiel mit 50 Elementen stellt ein besonderes Problem dar: Die
eine Teilmenge umfasst in der 2--dimensionalen Ebene teilweise die
restlichen Elemente. Siehe Abbildung \ref{umfassend}.

\begin{figure}[htbp]
%\[ \epsfxsize=4cm \epsfbox{umfassen.ps} \]
\caption{2 sich umfassende Mengen}
\label{umfassend}
\end{figure}

Die Zielfunktionen, die in den Clustern Distanzen summieren bzw. maximieren,
sind nicht in der Lage diese Form zu erkennen und weisen immer einen Teil
der "au"seren Menge der inneren zu. Lediglich eine Partitionierung in 3 
Cluster ist derart m"oglich, da"s die Elemente in der kleinen Teilmenge
einen eigenen Cluster bilden (siehe Abbildung \ref{umfassende}).

\begin{figure}[htbp]
%\[ \epsfxsize=4cm \epsfbox{um3farbe.ps} \]
\caption{Partitionierung in 3 Cluster}
\label{umfassende}
\end{figure}

Auch diese "`leichtere"' Partitionierung war nicht durch {\it eine}
Zielfunktion realisierbar, sondern es wurde hierzu die Funktion
 $3\cdot \sum\frac 1n \sum \ + \ \sum\max \ + \ \max\max$ verwendet.


\section{Ergebnisse}
Aus den in diesem Kapitel gewonnen Erkentnissen mu"s nun festgelegt 
werden, welche Verkn"upfungen aus Zielfunktion, Nachbarschaft und
Suchverfahren auf die medizinischen Daten angewandt werden soll.
Dazu werden die Eintr"age aus der Tabelle \ref{tabgleich} gestrichen, 
die nach den Ergebnissen aus \ref{kap42} keine Partitionierung erkennen
konnten. Wir erhalten dann als neue "Ubersicht von brauchbaren Kombinationen 
die Tabelle \ref{tabergebnisse}.

\begin{table}[htb]
\caption{Kombinationen von Zielfunktion, Nachbarschaft und Suchverfahren, 
\label{tabergebnisse}
die in den bisherigen Tests brauchbare Ergebnisse lieferten. Es bedeuten hierbei: NC = Nearest Cluster, C = Centroid, R = Random, SC = Simultaneous Centroid.}
\begin{center}
\begin{tabular}{l||c|c|c|c|c|c}
   & $\sum\sum$ & $\sum\sum^2$ & $\sum\frac 1n \sum$ & $\max\sum$ & $\sum\max$ & $\max\max$\\
\hline\hline
LS &\scriptsize{\begin{tabular}{c}NC\\C\\\\SC\end{tabular}}&
	\scriptsize{\begin{tabular}{c}NC\\C\\\\SC\end{tabular}}&
	\scriptsize{\begin{tabular}{c}\\\\\\SC\end{tabular}}&
	\scriptsize{\begin{tabular}{c}\\\\\\SC\end{tabular}}&
	\scriptsize{\begin{tabular}{c}\\C\\\\SC\end{tabular}}&
	\scriptsize{\begin{tabular}{c}\\C\\\\SC\end{tabular}}
	\\ \hline
II &\scriptsize{\begin{tabular}{c}NC\\C\\\\SC\end{tabular}}&
	\scriptsize{\begin{tabular}{c}NC\\C\\\\SC\end{tabular}}&
	\scriptsize{\begin{tabular}{c}\\\\\\SC\end{tabular}}&
	\scriptsize{\begin{tabular}{c}\\\\\\SC\end{tabular}}&
	\scriptsize{\begin{tabular}{c}\\\\\\SC\end{tabular}}&
	\scriptsize{\begin{tabular}{c}\\\\\\SC\end{tabular}}
	\\ \hline
SA &\scriptsize{\begin{tabular}{c}NC\\C\\R\\\quad\end{tabular}}&
	\scriptsize{\begin{tabular}{c}NC\\C\\R\\\quad\end{tabular}}&
	\scriptsize{\begin{tabular}{c}\\\\\\\quad\end{tabular}}&
	\scriptsize{\begin{tabular}{c}\\C\\\\\quad\end{tabular}}&
	\scriptsize{\begin{tabular}{c}\\C\\\\\quad\end{tabular}}&
	\scriptsize{\begin{tabular}{c}\\C\\\\\quad\end{tabular}}
	\\ \hline
SL &\scriptsize{\begin{tabular}{c}NC\\C\\R\\SC\end{tabular}}&
	\scriptsize{\begin{tabular}{c}NC\\C\\R\\SC\end{tabular}}&
	\scriptsize{\begin{tabular}{c}\\C\\\\SC\end{tabular}}&
	\scriptsize{\begin{tabular}{c}\\C\\\\SC\end{tabular}}&
	\scriptsize{\begin{tabular}{c}\\C\\\\SC\end{tabular}}&
	\scriptsize{\begin{tabular}{c}\\C\\\\SC\end{tabular}}
	\\ \hline
TA &\scriptsize{\begin{tabular}{c}NC\\C\\R\\\quad\end{tabular}}&
	\scriptsize{\begin{tabular}{c}NC\\C\\\\\quad\end{tabular}}&
	\scriptsize{\begin{tabular}{c}\\\\\\\quad\end{tabular}}&
	\scriptsize{\begin{tabular}{c}\\C\\\\\quad\end{tabular}}&
	\scriptsize{\begin{tabular}{c}\\C\\\\\quad\end{tabular}}&
	\scriptsize{\begin{tabular}{c}\\C\\\\\quad\end{tabular}}
	\\ \hline
TS &\scriptsize{\begin{tabular}{c}NC\\C\\\\\quad\end{tabular}}&
	\scriptsize{\begin{tabular}{c}NC\\C\\\\\quad\end{tabular}}&
	\scriptsize{\begin{tabular}{c}\\\\\quad\\\quad\end{tabular}}&
	\scriptsize{\begin{tabular}{c}\\\\\\\end{tabular}}&
	\scriptsize{\begin{tabular}{c}\\\\\\\end{tabular}}&
	\scriptsize{\begin{tabular}{c}\\\\\\\end{tabular}}
	\\ \hline\hline
CA & ja & ja & ja & & &
\end{tabular}
\end{center}
\end{table}

Alle Verfahren zeigen die Eigenschaft, da"s die Anzahl der Elemente pro
Cluster in der optimalen Partition etwa gleich gro"s ist. Dementsprechend
sollten die Erwartungen an die Eingabedaten gestellt werden: Wenn
mehrere Krankheitstypen in einer Elementmenge vermutet werden, so sollten
diese etwa gleichh"aufig auftreten.

Die {\bf Anzahl der Nachbarn} bei der Nearest--Cluster Nachbarschaft scheint in
einer Gr"o"senordnung von 5 Nachbarn gen"ugend gro"s zu sein, um optimale
L"osungen in einer gleichm"a"sig verteilten Elementmenge zu finden.
Wenn Strukturen in der Menge existieren, kann diese Zahl sogar noch weiter
gesenkt werden. Da wir vermutete Strukturen in einer medizinischen 
Datenbank aufsp"uren wollen, werden wir uns auf eine Nachbaranzahl von
maximal 5 betrachteten Nachbarn bei Nearest--Cluster beschr"anken.

Die anderen beiden {\bf Single Nachbarschaften} (Centroid, Random) lieferten
keine so guten Ergebnisse wie Nearest--Cluster; folglich werden wir
Nearest--Cluster als die einzige Single Nachbarschaft verwenden.

Bei den {\bf Simultaneous Nachbarschaften} konnte die Centroid Nachbarschaft
klar "uberzeugen. Auch diese wollen wir auf die medizinischen Daten
anwenden.

Da sich die {\bf Zielfunktionen} $\sum\sum$ und $\sum\sum^2$ nicht wesentlich 
voneinander unterscheiden,
wollen wir uns bei diesen beiden im weiteren auf $\sum\sum$ beschr"anken.
Die Maximierungszielfunktionen sind kein gutes Optimalit"atskriterium f"ur 
ein einfaches Austauschverfahren, weil zu selten der Zielfunktionswert
verbessert wird. Nur bei der Simultaneous Nachbarschaft waren diese
Funktionen in der Lage optimale Ergebnisse zu finden. 
Diese Zielfunktionen werden also nur unter Simultaneous--Centroid
angewendet, ebenso wie die Funktion $\sum\frac 1n \sum$.

Unter Verwendng der Nachbarschaft Simultaneous--Centroid k"onnen auch
sehr gro"se Datenmengen in kurzer Zeit optimal partitioniert werden,
da diese Nachbarschaft ein sehr gutes Konvergenzverhalten zeigt.

Der Rechenaufwand der {\bf Algorithmen} liegt in der Bestimmung des
Zielfunktionswertes, was sich gut erkennen l"a"st, wenn man statt einer
zwei Zielfunktionen bei der Berechnung aktiviert.

Die Tabu Suche lieferte h"aufig L"osungen, die eine geringere Clusteranzahl
besa"sen, als festgelegt. Dies wurde in der Implementierung abgefangen,
so da"s nur L"osungen mit der gew"unschten Clusteranzahl zul"assig sind.

Im folgenden Kapitel sollen alle Algorithmen angewandt werden,
wobei wir erwarten, da"s
Simulated Annealing / Localopt und Combined Alogorithms
die besten Ergebnisse liefern, wobei ersterer die l"angere
Rechenzeit beanspruchen wird.

Hier die "Ubersicht, welche Kombinationen
getestet werden sollen:

\begin{center}
\begin{tabular}{l||c|c|c|c|c|c}
   & $\sum\sum$ & $\sum\frac 1n \sum$ & $\max\sum$ & $\sum\max$ & $\max\max$\\
\hline\hline
LS &\scriptsize{\begin{tabular}{c}NC\\SC\end{tabular}}&
	\scriptsize{\begin{tabular}{c}\\SC\end{tabular}}&
	\scriptsize{\begin{tabular}{c}\\SC\end{tabular}}&
	\scriptsize{\begin{tabular}{c}\\SC\end{tabular}}&
	\scriptsize{\begin{tabular}{c}\\SC\end{tabular}}
	\\ \hline
II &\scriptsize{\begin{tabular}{c}NC\\SC\end{tabular}}&
	\scriptsize{\begin{tabular}{c}\\SC\end{tabular}}&
	\scriptsize{\begin{tabular}{c}\\SC\end{tabular}}&
	\scriptsize{\begin{tabular}{c}\\SC\end{tabular}}&
	\scriptsize{\begin{tabular}{c}\\SC\end{tabular}}
	\\ \hline
SA &\scriptsize{\begin{tabular}{c}NC\\\quad\end{tabular}}&
	\scriptsize{\begin{tabular}{c}\\\quad\end{tabular}}&
	\scriptsize{\begin{tabular}{c}C\\\quad\end{tabular}}&
	\scriptsize{\begin{tabular}{c}C\\\quad\end{tabular}}&
	\scriptsize{\begin{tabular}{c}C\\\quad\end{tabular}}
	\\ \hline
SL &\scriptsize{\begin{tabular}{c}NC\\SC\end{tabular}}&
	\scriptsize{\begin{tabular}{c}\\SC\end{tabular}}&
	\scriptsize{\begin{tabular}{c}C\\SC\end{tabular}}&
	\scriptsize{\begin{tabular}{c}C\\SC\end{tabular}}&
	\scriptsize{\begin{tabular}{c}C\\SC\end{tabular}}
	\\ \hline
TA &\scriptsize{\begin{tabular}{c}NC\\\quad\end{tabular}}&
	\scriptsize{\begin{tabular}{c}\\\quad\end{tabular}}&
	\scriptsize{\begin{tabular}{c}C\\\quad\end{tabular}}&
	\scriptsize{\begin{tabular}{c}C\\\quad\end{tabular}}&
	\scriptsize{\begin{tabular}{c}C\\\quad\end{tabular}}
	\\ \hline
TS &\scriptsize{\begin{tabular}{c}NC\\\quad\end{tabular}}&
	& & & 
	\\ \hline\hline
CA & ja & ja & ja & & &
\end{tabular}
\end{center}
