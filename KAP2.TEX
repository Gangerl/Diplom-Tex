\chapter{Lokale Suche}
\label{lokalesuche}
Dieses Kapitel gibt einen "Uberblick "uber die verschiedenen Verfahren
der lokalen Suche und deren Zusammenspiel mit den Zielfunktionen aus
Kapitel \ref{zielfunktionenkapitel}. Ausgehend von einer allgemeinen
Problemstellung und einem ersten Algorithmus werden immer weiter
verfeinerte Algorithmen entwickelt, die sich der an sie gestellten
Aufgabe optimal anpassen sollen.

%\section{Grundlagen}
\section{Situation}
Ein diskretes Optimierungsproblem wird wie folgt beschrieben:
Aus der endlichen Menge $S$ der zul"assigen L"osungen suchen wir 
unter einer Zielfunktion $c:S \longmapsto \R$ diejenige L"osung
$s^*$, f"ur die gilt
\begin{equation}
	c(s^*) \ \leq\  c(s) \quad \forall\, s \in S.
\end{equation}

Lokale Suche ist nun ein Verfahren, um schwierige Probleme
in angemessener Zeit l"osen zu k"onnen, wobei man darauf verzichtet, 
eine optimale L"osung zu erhalten.
Daf"ur beginnt man mit einer Startl"osung $s_0$, konstruiert 
"`benachbarte"' L"osungen und sucht sich unter diesen Nachbarn eine
bez"uglich der Zielfunktion bessere L"osung als die bisherige und wendet
das Verfahren erneut an, bis man eine L"osung gefunden hat, die keine
besseren Nachbarn mehr besitzt. Dabei hei"st eine L"osung $t \in S$ {\em 
benachbart} zu $s$, wenn wir $t$ von $s$ aus in einem Schritt erreichen.
$N(s) = \{ t \,|\, t$ ist Nachbar von $s \}$ hei"st dann {\em Nachbarschaft}
von $s$.

\section{Nachbarschaften}
\label{nachbarn}
In der Klassifikationstheorie bietet sich folgende Nachbarschaft an:
In einer gegebenen L"osung verschiebe ein Element $x_j$ aus einen Cluster
in einen anderen und erhalte somit eine Nachbarl"osung.

\begin{figure}[htp]
\[
\input b3-1.pic
\]
\caption{Nachbarl"osung}
\end{figure}

Wie der andere Cluster zu w"ahlen ist, stellt uns nat"urlich
vor das Problem "`Finde einen geeigneten Cluster f"ur den Punkt $x_j$"'.
Hierf"ur bieten sich mehrere Vorgehensweisen an:
\begin{enumerate}
\item W"ahle den Cluster aus, aus dem ein Punkt zu $x_j$ am n"achsten liegt
	({\em Nearest--Cluster Nachbarschaft}).
\item W"ahle den Cluster aus, dessen Schwerpunkt (Centroid) dem Punkt
	$x_j$ am n"achsten liegt ({\em Centroid Nachbarschaft}).
\item W"ahle einen Cluster zuf"allig aus ({\em Random Nachbarschaft}).
\end{enumerate}
Diesen Verfahren ist nat"urlich gemeinsam, da"s der neue Cluster
ungleich dem sein soll, in dem $x_j$ derzeit liegt.
Das Random--Verfahren dient hierbei nur zu Vergleichszwecken, zum einen
um zu testen, wie gut die Algorithmen auf unstrukturierten Nachbarschaften
arbeiten, zum anderen um die anderen Verfahren besser einsch"atzen zu
k"onnen.
Das zweite Verfahren, Centroid Nachbarschaft, wird in der Literatur
h"aufig verwendet, jedoch in einer spezielleren Form, auf die wir sp"ater
noch zur"uckkommen wollen. Unser Hauptaugenmerk wollen wollen wir also
auf die Nearest--Cluster Nachbarschaft richten.

\begin{figure}[htp]
\[\psfig{figure=b3-2}\]
\caption{Nearest--Cluster Nachbarschaft: W"ahle den Cluster aus, in
	dem ein Punkt der n"achste ist.}
\end{figure}

Bei einer gleichm"a"sigverteilten Anfangsl"osung beginnt der Proze"s der
Clusterbildung recht langsam, wie Abbildung \ref{b3-3} zeigt.
\begin{figure}[htp]
\[\psfig{figure=b3-3}\]
\caption{Gleichm"a"sige Startl"osung und m"ogliche Anordnung nach 3
	Rechenschritten \label{b3-3}}
\end{figure}

Wenn sich aber bereits Cluster gebildet haben, werden h"ochstens noch
die Randpunkte verschoben (Abbildung \ref{b3-4}).

\begin{figure}[htbp]
\[\psfig{figure=b3-4}\]
\caption{m"oglicher weiterer Ablauf \label{b3-4}}
\end{figure}

Das Auf\/finden eines Nachbarclusters ist nat"urlich viel zu aufwendig,
wenn daf"ur die gesamte Punktmenge durchlaufen werden mu"s.
Wie man sich an Abbildung \ref{b3-5} 
verdeutlichen kann, ist es sicher nicht n"otig f"ur den mittleren Punkt des
Clusters mit den Kreisen einen neuen Cluster zu suchen
(oder f"ur den Punkt links unten in den 
Abbildungen \ref{b3-3} und \ref{b3-4}), 
da er in seiner Umgebung zuviele Punkte aus dem eigenen Cluster besitzt.
Dementsprechend gibt man sich eine Zahl $t$ vor, die die Anzahl der
betrachteten Nachbarpunkte beschr"ankt.
Diese N"achstenachbarn--Anzahl ist sicherlich von der Anzahl der zu 
betrachtenden Elemente und ihrer Verteilung abh"angig. Wir werden
versuchen, einen optimalen Wert f"ur diese Zahl zu bestimmen.

\begin{figure}[htbp]
\[ \psfig{figure=b3-5} \]
\caption{Der mittlere Punkt im Kreis--Cluster braucht derzeit sicher keinem
anderen Cluster zugeordnet werden. \label{b3-5}}
\end{figure}

Wir bekommen also folgende Prozedur, die uns aus einer gegebenen L"osung
und einem gegebenen Punkt eine Nachbarl"osung konstruiert,  und die true
zur"uckliefert, wenn sie eine gefunden hat, sonst false.
Zur Beschreibung dieser und folgender Prozeduren und Algorithmen 
verwenden wir "`Pseudo--Pascal"', 
das hei"st Schleifen, Bl"ocke und Bedingungskonstrukte aus 
der Programmiersprache {\sc Pascal} vermischt mit
mathematischen Formulierungen.

\begin{algorithm*}
PROCEDURE Nearest--Cluster : BOOLEAN;\\
BEGIN
\begin{Block}
	W"ahle zu dem gegebenen Punkt den besten Nachbarcluster NC, ungleich dem 
	aktuellen Cluster von x, aus den n"achsten $t$ Punkten von x aus;\\
	IF NC gefunden THEN BEGIN
	\begin{Block}
		Verschiebe $x$ in den Cluster NC;\\
		RETURN true;
	\end{Block}
	END;\\
	RETURN false;
\end{Block}
END;
\end{algorithm*}


Die  Random Nachbarschafts Prozedur wird nicht aufgef"uhrt;
sie w"ahlt einfach zuf"allig einen Cluster aus und
l"a"st sich sehr leicht nach obigem Vorbild konstruieren.

Es kommt hierbei nat"urlich vor, da"s f"ur den einen oder anderen Punkt
kein Cluster gefunden wird. Darauf m"ussen die Algorithmen nat"urlich
entsprechend reagieren, z.B. indem sie versuchen andere Punkte zu 
verschieben. Somit erhalten wir ein Abbruchkriterium: Wenn f"ur keinen
Punkt ein neuer Cluster unter seinen $t$ n"achsten Nachbarn zu
finden war, dann STOP.

Man kann noch ein weitergehendes System der Nachbarschaftsbildung 
konstruieren: Verschiebe nicht immer nur einen Punkt und betrachte dieses
dann als Nachbarn, dessen Zielfunktionswert man betrachtet, 
sondern verschiebe {\it alle} Punkte zu je einem m"oglichen neuen Cluster!

Diese Methode l"a"st sich auf Nearest--Cluster Nachbarschaft
anwenden: Suche f"ur alle Punkte zugleich einen neuen Cluster.
Dieses Verfahren wollen wir dann mit 
{\em Simultaneous--Nearest--Cluster Nachbarschaft} bezeichnen.

Diese Vorgehensweise wird von den meisten Autoren f"ur die
Centroid Nachbarschaft verwendet, wir wollen hier aber, damit die 
Begriffe nicht durcheinandergeraten, dementsprechend auch von
{\em Simultaneous--Centroid Nachbarschaft} sprechen:
In jedem Iterationsschritt werden die Gruppencentroide berechnet und alle
Punkte simultan in den Cluster verschoben, dessen Centroid ihnen am 
n"achsten liegt. Somit bekommen wir jeweils nur noch genau einen 
m"oglichen Nachbarn und k"onnen nicht mehr einen zuf"alligen oder
besten ausw"ahlen. Das hei"st, die Algorithmen m"ussen dementsprechend
angepasst werden.

\begin{algorithm*}
PROCEDURE Simultaneous--Centroid : BOOLEAN;\\
BEGIN
\begin{Block}
	success := false;\\
	Berechne die Gruppencentroide;\\
	FOR all Punkte DO BEGIN
	\begin{Block}
		Verschiebe, falls n"otig, den Punkt in den Cluster mit
		dem n"achsten Gruppencentroid;\\
		IF es wurde ein neuer Cluster gefunden THEN success := true;
	\end{Block}
	END;\\
	RETURN success;
\end{Block}
END;
\end{algorithm*}

F"ur die Nearest--Cluster Nachbarschaft kann die Prozedur nat"urlich
analog formuliert werden. Die R"uckgabe eines Wahrheitswertes, der 
aussagt ob ein Punkt verschoben wurde, erleichtert den Abbruch des
eigentlichen Verfahrens, wenn es keine Ver"anderungen an der L"osung
gegeben hat (man spart so eine Berechnung des Zielfunktionswertes).

Eine Anwendung auf die Random Nachbarschaft scheint nicht erfolgsversprechend, 
denn wenn jedesmal alle Punkte einem zuf"alligen Cluster zugeordnet 
werden, kann dies nicht vern"unftig sein.

Das Verhalten der Simultaneous--Nearest--Cluster Nachbarschaft l"a"st sich
nicht so leicht einsch"atzen: Bei bereits existierenden H"aufungen erfolgt
die Nachbarauwahl sicher sehr gut. Jedoch wird bei 2 Elementen, die eine
geringe Distanz zueinander aufweisen und in verschiedenen Clustern liegen, 
ein Austausch stattfinden. Dementsprechend erwarten wir von der 
Simultaneous--Centroid Nachbarschaft die besseren Ergebnisse.

Um sp"ater die Nearest--Cluster Nachbarschaft besser mit der 
Simul\-taneous--Centro\-id Nachbarschaft
vergleichen zu k"onnen, entwickeln wir auch noch das naheliegende
Verfahren {\em Cent\-roid Nachbarschaft} bei dem auch zu nur 
einem vorgegebenen Punkt $x$ ein neuer Cluster gesucht wird.
Auch diese Routine kann analog zu obiger Prozedur Nearest--Cluster
bestimmt werden.

\section{Algorithmen}
\label{algorithmeneins}

Aus der Nachbarschaftsidee lassen sich nun verschiedene Algorithmen
konstruieren. Der grundlegende, auf den alle weiteren aufgebaut sind,
ist folgender:

\begin{algorithm}{Local Search}
W"ahle eine Startl"osung $s \in S$;\\
REPEAT
\begin{Block}
	W"ahle einen Nachbarn $s' \in N(s)$ mit $c(s') < c(s)$;
\end{Block}
UNTIL Stopkriterium;\\
best $ := c(s)$;\\
$s^\ast := s$;
\end{algorithm}

Als Stopkriterium w"ahlt man i.a. eine Bedingung wie $c(s') \geq c(s)
\ \forall\, s' \in N(s)$.
Somit ist die Terminierung des Algorithmuses gesichert, da  wir nur mit einer
L"osung, die einen kleineren Funktionswert als die vorherige besitzt,
einen neuen Schleifendurchlauf beginnen, und wir somit nicht zu 
bereits besuchten L"osungen zur"uckkehren.
Da die Anzahl der m"oglichen Partitionen endlich ist, mu"s der
Algorithmus terminieren.
Dieser Algorithmus kann verschiedene L"osungen $s^*$ liefern, da er leicht
in sogenannten {\em lokalen Optima} endet und das globale Minimum nur
zuf"allig erreichen kann. 

Bei der Nachbarauswahl haben wir im vorherigen Abschnitt gesehen, 
da"s man nicht zu allen Punkten einen Nachbarn bestimmen kann.
Wenn also nun
die Nachbarschaftsprozedur keinen neuen Cluster zur"uckliefern
kann, z.B. weil die Nearest--Cluster Nachbarschaft keinen Punkt eines
anderen Clusters in der Umgebung unseres Punktes gefunden hat, oder weil die
Centroid Nachbarschaft den gleichen Cluster zur"uckliefert, dann 
m"ussen die anderen Punkte der Punktmenge getestet werden. Wenn kein
Punkt mehr eine Verbesserung des Zielfunktionswertes verspricht,
terminiert das Verfahren.

Um jetzt alle Punkte abarbeiten zu k"onnen, ben"otigen wir eine
geordnete Liste, und damit wir diese nicht immer von vorn durchlaufen
m"ussen, brauchen wir einen zuf"alligen Anfangspukt in dieser Liste.
Die Liste l"a"st sich auch einfach "uber die Indizes der Elemente
darstellen.

Der folgende Algorithmus benutzt als Nachbarschaft
Nea\-rest--Cluster, es k"onn\-te aber an der entsprechenden
Stelle genausogut Cent\-roid oder Ran\-dom stehen.

\begin{algorithm}{Local Search --- Modified}
\label{localopt}
\label{listenalgo}
W"ahle eine Startl"osung $s \in S$;\\
REPEAT
\begin{Block}
	point := Rand(1,anzahl);\\
	n := 0;\\
	success := false;\\
	REPEAT
	\begin{Block}
		point := (point + n) modulo (anzahl+1);\\
		IF point = 0 THEN INC(point);\\
		$s' := s$;\\
		IF Nearest--Cluster($s'$, point) THEN
		\begin{Block}
			IF $c(s') < c(s)$ THEN BEGIN
			\begin{Block}
				$s := s'$;\\
				success := true;
			\end{Block}
			END;
		\end{Block}
		INC(n);
	\end{Block}
	UNTIL (n $>$ anzahl) oder success;
\end{Block}
UNTIL n $>$ anzahl;\\
best $ := c(s)$;\\
$s^\ast := s$;
\end{algorithm}

Eine eigene Prozedur f"ur den Programmteil, der "uberpr"uft, ob ein 
Nachbar aus diesem Punkt konstruiert werden kann, w"are w"unschenswert.
Da aber die Aufgaben einer solchen Routine dem jeweiligen Algorithmus
angepasst w"aren, und somit eine solche Prozedur zu genau einem 
Algorithmus geh"orte, kann man darauf auch verzichten.

Eine Verbesserung der Konvergenzeigenschaft wird erreicht, indem man  nicht
mit einem beliebigen besseren Nachbarn, sondern mir dem besten Nachbarn
einen neuen Iterationsschritt beginnt:

\begin{algorithm}{Iterative Improvement}
\label{iterativeimprovement}
W"ahle eine Startl"osung $s \in S$;\\
{REPEAT}
\begin{Block}
	W"ahle den Nachbarn $s'$ von $s$ in $N(s)$ mit kleinstem 
	Zielfunktionswert aus;\\
	{IF} $c(s') < c(s)$ {THEN} $s := s'$;
\end{Block}
{UNTIL} $c(s') \geq c(s)$;\\
best := $c(s)$;\\
$s^* := s$;
\end{algorithm}

Jedoch erreichen wir auch bei diesem Verfahren nur lokale Optima,
i.a.\ ein wenig schneller. 

Nun wollen wir noch den modifizierten Algorithmus 
vorstellen, der auf einer Simultaneous Nachbarschaft arbeitet.
Er arbeitet f"ur die Verfahren Local Search und Iterative Improvement
gleich, da wir nicht mehr einen bestimmten Nachbarn ausw"ahlen k"onnen.

\begin{algorithm}{Simultaneous}
\label{simultaneous}
W"ahle eine Startl"osung $s \in S$;\\
REPEAT
\begin{Block}
	Bestimme den Nachbarn $s'$ von $s$ gem"a"s der
	Simultaneous Nachbarschaft;
\end{Block}
UNTIL Stopkriterium;\\
best $ := c(s)$;\\
$s^\ast := s$;
\end{algorithm}

F"ur dieses Verfahren gilt folgender

\begin{satz}
F"ur die durch Algorithmus \ref{simultaneous} unter der Nachbarschaft Centroid
konstruierte Folge von Gruppierungen
\[ \G^1, \G^2,\dots,\G^r,\G^{r+1},\dots\qquad \mbox{mit } \G^r := \{
	C_1^r,\dots,C_k^r\} \]
gilt unter der Distanznorm $||\cdot||$
\[ Z_1(\G^1) \geq Z_1(\G^2) \geq \cdots \geq Z_1(\G^r) \geq Z_1(\G^{r+1})
	\geq  \cdots . \]
$Z_1$ ist hierbei wieder das Varianzkriterium aus \ref{zielfunktionenkapitel},
das "aquivalent zu $F_3$ ist.
\end{satz}

{\bf Beweis:}
\vspace{-3mm}
\begin{eqnarray*}
Z_1(\G^r) & = & 
			 \sum_{l=1}^k \sum_{i \in C_l^r} || x_i - \bar x _l^r ||^2\\
	& \geq & \sum_{l=1}^k \sum_{i \in C_l^r} \min_j || x_i - \bar x _j^r ||^2\\
	& =    & \sum_{l=1}^k \sum_{i \in C_l^{r+1}} || x_i - \bar x _l^r ||^2\\
	& \geq & \sum_{l=1}^k \sum_{i \in C_l^{r+1}} || x_i - \bar x _l^{r+1} ||^2\\
	& =    & Z_1(\G^{r+1})\qquad\Box
\end{eqnarray*}
\vspace{5mm}

Das hei"st eine im $(r+1)$--ten Schritt gewonnene Gruppierung liefert den
gleichen oder einen besseren Wert der Zielfunktion als die im $r$-ten
Schritt. 2 F"alle sind denkbar:

Fall 1: Mehrere Gruppierungen reproduzieren sich in periodischer Reihenfolge.

Dieser Fall ist nicht auszuschlie"sen, aber mehr theoretisch und kommt in 
der Praxis selten vor.

Fall 2: Das Verfahren bricht ab, da zwei aufeinanderfolgende Gruppierungen
gleich sind.

In diesem Fall hat man wieder ein lokales Minimum gefunden.

Da der erste Fall "au"serst selten ist, kann man das Stopkriterium 
formulieren als $s = s'$, oder besser als $c(s') \geq c(s)$.

\begin{figure}[htbp]
\[ \input b2-2.pstex_t \]
\caption{M"ogliche L"osungsverteilung \label{bergundtal}}
\end{figure}

Wie man sich an der Abbildung \ref{bergundtal} verdeutlichen kann, 
braucht man also eine
Nachbarschaftsauswahl, die auch Spr"unge "uber lokale Berge zul"a"st,
damit man bessere Optima finden kann.

Um also zu vermeiden, da"s der Algorithmus in einem lokalen Optimum terminiert, 
da die Nachbarauswahl keine anderen L"osungen mehr zul"a"st, gibt es
ein mehr zufallsgesteuertes Verfahren, genannt {\em Simulated
Annealing}.
Hierbei beginnt man einen Iterationsschritt 
auch mit solchen Nachbarl"osungen, die den 
Zielfunktionswert zwar nicht verbessern, die aber zumindest in einer
gewissen N"ahe liegen. Diese ist abh"angig von der Laufzeit des
Algorithmus und wird automatisch gesteuert. Das hei"st:
Nach Wahl einer Startl"osung akzeptiert man in jeder Iteration eine
Nachbarl"osung $s'$ mit der Wahrscheinlichkeit
\[
	\min \{1, \exp(-\frac{c(s') - c(s)}{c_i}) \},
\]
wobei die $c_i$ eine monoton fallende Folge mit $\lim_{i \to \infty}
c_i = 0$ sind.
Der $c_{i+1}$--Wert wird im allgemeinen durch eine Funktion $g(c_i)$ 
gegeben.
Eine Betrachtung obiger Wahrscheinlichkeit zeigt, da"s eine bessere
Nachbarl"osung immer, eine schlechtere dagegen eher zu Beginn des 
Verfahrens akzeptiert wird, und ein Verlassen eines lokalen Minimums
zul"a"st.
Als Nachbarschaft kommen die Simul\-taneous Versionen nicht in 
Frage, da wir bei ihnen nur "`straight forward"' gehen, aber
nicht unter verschiedenen Nachbarn w"ahlen k"onnen.

Eine Schwierigkeit ist nun, wie dieses Verfahren beendet werden soll.
Nach einer gewissen Laufzeit wird es immer unwahrscheinlicher, da"s 
noch gute L"osungen gefunden werden, aber m"oglich ist es nat"urlich.

Im folgenden Algorithmus liefere $rand(0,1)$ eine gleichverteilte
Zufallszahl zwischen 0 und 1.

\begin{algorithm}{Simulated Annealing}
i := 0;\\
Initialisiere $c_0$;\\
W"ahle eine Startl"osung $s \in S$;\\
best := $c(s)$;\\
$s^* := s$;\\
{REPEAT}
\begin{Block}
	W"ahle eine zuf"allige L"osung $s' \in N(s)$;\\
	{IF} $rand(0,1) < \min \{1,\exp({\scriptstyle - \frac{c(s') - c(s)}
	{c_i}})\}$ {THEN} $s := s'$;\\
	{IF} $c(s') < $ best {THEN BEGIN}
	\begin{Block}
		$s^* := s'$;\\
		best := $c(s')$;
	\end{Block}
	{END};\\
	$c_{i+1} := g(c_i)$;\\
	i := i + 1;
\end{Block}
{UNTIL} Stopkriterium;
\end{algorithm}

Es kann gezeigt werden, da"s unter bestimmten Bedingungen, die in der
Praxis normalerweise erf"ullt sind, dieser Algorithmus asymptotisch
konvergent ist (Hajek [1988]).
Der Wert $c_0$ sollte anfangs gro"s genug gew"ahlt werden, damit zu
Beginn die Akzeptanz--Wahrscheinlichkeit nahe bei 1 liegt. Eine
gebr"auchliche Funktion f"ur $g(c_i)$ ist $c_{i+1} = \alpha c_i$,
wobei ein Wert f"ur $\alpha$ typischerweise aus dem Intervall 
$[0.8,0.99]$ gew"ahlt wird.
Mehr "uber die Kontrollfunktion $g$ und das Stopkriterium findet man in
Van Laarhoven und Aarts [1987]; eine M"oglichkeit den Algorithmus
zu terminieren ist ihn nach einer bestimmten Laufzeit zu stoppen, 
wie es auch das Programm, das in Kapitel \ref{handbuch} beschrieben
ist, erm"oglicht.
Eine andere Methode ist, den Algorithmus zu beenden, wenn $c_i$ kleiner
als eine festgelegt Schranke geworden ist.
Ein drittes Abbruchkriterium liegt vor, wenn kein Nachbar mehr gefunden wurde,
wie es bei unseren Nachbarschaften geschehen kann. Das hei"st wir
verwenden analog zu Algorithmus \ref{listenalgo} auch hier eine Liste,
die "`zuf"allig"' "uber die Punkte f"uhrt, und die beim Erreichen des 
Endes der Liste den Algorithmus terminiert.
Eine weitere M"oglichkeit des Abbruchs liegt vor, wenn in einer Zeitspanne
von einer gewissen Anzahl von Schritten keine Verbesserung stattgefunden hat.

Nach einer Idee von Brucker et al. [1993] kann man Simulated Annealing
auch mit einem der vorhergehenden Verfahren Local Search oder besser
Iterative Improvement kombinieren:
F"ur das verwendete Verfahren (Local Search oder Iterative Improvement)
gebrauchen wir in diesem Zusammenhang die Bezeichnung {\em Localopt}.
Ausgehend von einer L"osung, die Localopt
liefert, w"ahlt man in der REPEAT\dots UNTIL--Schleife von
Simulated Annealing nicht irgendeinen Nachbarn aus, sondern
man wendet auf diesen Nachbarn erst Localopt an. 
Wir bekommen dann den

\begin{algorithm}{Simulated Annealing / Localopt}
i := 0;\\
Initialisiere $c_0$;\\
Wende Localopt an und erhalte somit eine Startl"osung $s$;\\
best := $c(s)$;\\
$s^* := s$;\\
{REPEAT}
\begin{Block}
	W"ahle eine zuf"allige L"osung $s' \in N(s)$;\\
	Wende auf $s'$ Localopt an;\\
	{IF} $rand(0,1) < \min \{1,\exp({\scriptstyle - \frac{c(s') - c(s)}
	{c_i}})\}$ {THEN} $s := s'$;\\
	{IF} $c(s') < $ best {THEN BEGIN}
	\begin{Block}
		$s^* := s'$;\\
		best := $c(s')$;
	\end{Block}
	{END};\\
	$c_{i+1} := g(c_i)$;\\
	i := i + 1;
\end{Block}
{UNTIL} Stopkriterium;
\end{algorithm}

Hierbei ist es auch m"oglich verschiedene Nachbarschaften einzusetzen:
Man benutzt z.B. Nearest--Clu\-ster in Simu\-lated An\-nea\-ling und
Simul\-taneous--Nea\-rest--Clu\-ster in der Local\-opt--Prozedur.

Eine kleine Ver"anderung der Akzeptanz--Regel bei Simulated Annealing
f"uhrt zu dem Verfahren {\em Threshold Acceptance}, bei dem
eine Nachbarl"osung dann akzeptiert wird, wenn die Differenz
$c(s') - c(s) < t_i$ ist, wobei $t_i$ wieder eine monoton fallende, 
nichtnegative Folge ist:

\begin{algorithm}{Threshold Acceptance}
i := 0;\\
Initialisiere $t_0$;\\
W"ahle eine Startl"osung $s \in S$;\\
best := $c(s)$;\\
$s^* := s$;\\
{REPEAT}
\begin{Block}
	W"ahle eine zuf"allige L"osung $s' \in N(s)$;\\
	%{\bf IF} $rand(0,1) < \min \{1,\exp({\scriptstyle - \frac{c(s') - c(s)}
	%{c_i}})\}$ {\bf THEN} $s := s'$;\\
	{IF} $c(s') - c(s) < t_i$ {THEN} $s := s'$;\\
	{IF} $c(s') < $ best {THEN BEGIN}
	\begin{Block}
		$s^* := s'$;\\
		best := $c(s')$;
	\end{Block}
	{END};\\
	$t_{i+1} := g(t_i)$;\\
	i := i + 1;
\end{Block}
{UNTIL} Stopkriterium;
\end{algorithm}

"Ahnlich wie Simulated Annealing liefert dieses Verfahren bei sehr langen
Rechenzeiten gute Ergebnisse.
Nachteile dieser drei zuletzt genannten Verfahren sind einmal das
erw"ahnte Terminierungsproblem, und zum anderen die Gefahr, zu 
bereits besuchten L"osungen zur"uckzukehren. 
Diese R"uckkehr kann positiv sein, wenn man von dort aus zu besseren
L"osungen als im ersten Durchlauf kommt, aber auch negativ, wenn
l"angere Zyklen entstehen.
Einen Ausweg hieraus bietet das Verfahren {\em Tabu Suche}:
% auf das 
%jedoch im Rahmen dieser Arbeit nicht n"aher eingegangen wird,
%da die bisher vorgestellten Algorithmen sehr gute L"osungen liefern,
%wie wir in Kapitel \ref{auswertungen} sehen werden, und Tabu Suche
%einen hohen Verwaltungsaufwand erfordert.

Bereits besuchte L"osungen werden in einer sogenannten {\em Tabu Liste}
gespeichert, und es werden nur solche L"osungen akzeptiert, die
sich nicht in der Liste befinden. Nat"urlich ist es viel zu aufwendig,
alle besuchten L"osungen zu speichern und jedesmal eine m"ogliche
neue L"osung mit allen in der Tabu Liste befindlichen zu vergleichen.
Aus diesem Grund speichert man bei einer neuen L"osung nur die 
speziellen Merkmale in der Liste ab. Dann sind alle die L"osungen
{\em tabu}, die ein Merkmal aus der Tabu Liste aufweisen. 
Wenn nun die Tabu Liste eine L"ange $t$ hat, k"onnen keine Zyklen
mit L"ange $\geq t$ auftreten.

F"ur unser Partitionierungsproblem bietet es sich nun an, in der
Tabu Liste den zuletzt verschobenen Punkt zu speichern; das hei"st
es empfehlen sich nur die nicht 
Simul\-taneous Nachbarschaften, da sich bei diesen
die Nachbarl"osungen zu sehr von ihrer Ausgangsl"osung unterscheiden.

Die L"ange der Tabu Liste kann somit h"ochstens gleich der Anzahl der
Elemente $-1$ sein, besser ist es nat"urlich deutlich darunter zu 
bleiben, sagen wir die L"ange betrage $r\%$ der Elementanzahl.
F"ur die Zahl $r$ werden wir durch Testreihen versuchen einen 
optimalen Wert zu bestimmen. Die Tabu Liste wird bei dieser
Anwendungsform nat"urlich zyklisch durchlaufen.

Neben der Tabu Liste gibt es noch das sogenannte {\em Aspiration
Criterion}. Eine L"osung, die eigentlich tabu ist, kann dann doch als
neue L"osung akzeptiert werden, wenn sie das Aspiration Criterion
erf"ullt. Zum Beispiel kann dies der Fall sein, wenn  die neue L"osung
um einen gewissen Faktor $k$ besser ist als die alte.

\begin{algorithm}{Tabu Search}
W"ahle eine Startl"osung $s \in S$;\\
best := $c(s)$;\\
$s^* := s$;\\
Tabu--Liste := $\emptyset$;\\
{REPEAT}
\begin{Block}
	$Cand(s) := \{ s' \in N(s) | $ Der "Ubergang zu $s'$ ist nicht tabu,
	oder $s'$ erf"ullt das Aspiration Criterion$\}$;\\
	W"ahle eine L"osung $\bar s \in Cand(s)$;\\
	Tabu--Liste := Tabu--Liste $\cup\  \{ \bar s \}$;\\
	$s := \bar s$;\\
	IF $c(s) < $ best THEN BEGIN
	\begin{Block}
		$s^\ast := s$;\\
		best := $c(s)$;
	\end{Block}
	END;
\end{Block}
UNTIL Stopkriterium;
\end{algorithm}

Jetzt ist noch festzulegen, wie die L"osung $\bar s$ aus der Menge 
$Cand(s)$ zu w"ahlen ist. Eine einfache Methode w"are: W"ahle $\bar s$ mit
$c(\bar s) = \min \{ c(s') | s' \in Cand(s) \}$. Dies ist allerdings f"ur
gro"se Probleme ($n \geq 50$) in angemessener Zeit nicht zu realisieren.
Anstelle dessen treffen wir folgendes Auswahlkriterium: W"ahle aus einem
nicht zu gro"sen Prozentsatz der L"osungen in $Cand(s)$ die beste aus.

F"ur das Stopkriterium gelten die Aussagen von Simulated Annealing.

Es hei"st, da"s Tabu Suche i.a. sehr schnell ist --- wir werden versuchen
diese Aussage zu best"atigen.

Wir erhalten somit folgende M"oglichkeiten, den Algorithmus und
die Nachbarschaft auszuw"ahlen:

\begin{table}[htp]
\begin{center}
\begin{tabular}{|l|c|c|}
\hline
& \multicolumn{2}{c|}{Nachbarschaft}\\
 \raisebox{1.5ex}[1.5ex]{Algorithmus} & Single & Simultaneous \\
 \hline
 Local Search & ja & ja \\
 Iterative Improvement & ja & wie Local Search \\
 Simulated Annealing & ja & nein\\
 SA / Localopt & SA und Localopt & nur Localopt\\
 Threshold Acceptance & ja & nein\\
 Tabu Search & ja & nein\\
 \hline
\end{tabular}
\end{center}
\caption{Kombinierbare Algorithmen und Nachbarschaften}
\end{table}

Es bleibt anzumerken, da"s bei all diesen Algorithmen auch L"osungen mit
einem oder mehreren leeren Clustern entstehen k"onnen, wenn im Laufe
des Verfahrens nach und nach alle Elemente aus diesen Clustern anderen
zugewiesen werden, ohne da"s neue hinzukommen. Das Programm \Clustering\
verhindert dies, indem es Partitionen mit leeren Clustern einen sehr gro"sen
Zielfunktionswert zuordnet, so da"s diese nicht eine akzeptable
Nachbarl"osung sein k"onnen. Es bleiben dann in der Regel
einpunktige Cluster zur"uck. Man sollte sich beim Auftreten dieses Falles
Gedanken machen, ob die gew"ahlte Clusteranzahl dem Problem entspricht.

\section{Startl"osungen}
Um eine Startl"osung zu erhalten, ben"otigt man eine Heuristik.
Je nachdem wie gut die Anfangspartition sein soll, mu"s hierf"ur
mit mehr oder weniger Aufwand gerechnet werden.
Eine m"ogliche Anfangsl"osung ist zum Beispiel die Elemente
per Zufallsfunktion in Cluster zu verteilen; eine andere sie einfach
der Reihe nach den Clustern zuzuordnen.

F"ur den Verlauf des Algorithmus scheint es geschickt zu sein,
mit einer m"oglichst guten L"osung zu starten, 
um dadurch die Laufzeit zu senken.
Andererseits kann es so eher geschehen, da"s man in einem lokalen 
Optimum endet, 
w"ahrend man mit einer zuf"alligen Anfangspartition sicher bessere
Chancen hat, die globale L"osung zu finden.
Dies gilt vor allem dann, wenn der Algorithmus mehrfach auf das
Problem angewandt wird, und man die L"osungen vergleichen kann.


\section{Verhalten der Nachbarschaften}
\subsection*{Single Nachbarschaften}

Man kann davon ausgehen, da"s 
unter einer der Maximierungs--Zielfunktionen $\max\max$, $\max \sum$,
$\sum\max$
die Algorithmen mit Single Nachbarschaften 
sehr schnell terminieren, ohne
eine gute L"osung gefunden zu haben. Denn wenn durch die 
Nachbarschaften bedingt eine Nachbarl"osung gew"ahlt wird, die das
aktuelle Maximum nicht verringert, so wird das Verfahren beendet.
Folgendes Beispiel zeigt dies, wenn als Zielfunktion $\max\max$
gew"ahlt ist.

{\bf Beispiel:}
Ausgangsl"osung sei

\begin{center}
\input b2-1.pic
\end{center}

bei der die Distanzen zwischen den 3 Quadraten gleich und 
betragsm"a"sig die gr"o"sten sein sollen. 
Der Kreis steht f"ur beliebig viele Elemente in anderen Clustern.
Der Zielfunktionswert wird also im Cluster mit den Quadraten angenommen.
Wird ein Element aus dem Kreis dem Cluster der Quadrate zugeordnet, 
so "andert sich an dem Wert der Zielfunkton nichts; andererseits, 
wenn ein Quadrat einem anderen Cluster zugeordnet wird, ver"andert
dies den Zielfunktionswert auch nicht, da noch immer zwei Elemente mit
maximaler Distanz in ein und demselben Cluster liegen.

\subsection*{Simultaneous Nachbarschaften}
\label{combined}
Obiges Problem kann bei den Simultaneous Nachbarschaften nicht auftreten,
da der neue Zielfunktionswert erst dann bestimmt wird, wenn alle Elemente
betrachtet und evtl. verschoben worden sind. Wenn jedoch nur noch sehr wenige
Elemente einen  neuen Cluster finden, kann es auch hier zu einer
fr"uhzeitigen Terminierung kommen. Da wir haupts"achlich die
Simultaneous--Centroid Nachbarschaft betrachten entstehen zus"atzliche
Schwierigkeiten, da Elemente, die in der Randzone eines Clusters in der
N"ahe eines anderen Clusters liegen, nicht die M"oglichkeit bekommen, 
ihre Zugeh"origkeit zu wechseln.

Aufgrund dieser Tatsachen scheint folgender Algorithmus erfolgsversprechend:

\begin{algorithm}{Combined Algorithms}
	1. Wende Algorithmus \ref{simultaneous} --- Simultaneous --- mit der
	Centroid Nachbarschaft an.\\
	\\
	2. Wende auf diese L"osung noch Algorithmus \ref{localopt} --- 
	modifizierte Lokale Suche --- an, der die Partitionierung aus
	Schritt 1 als Startl"osung erh"alt und die Nachbarschaft
	Nearest--Cluster verwendet.
\end{algorithm}

Die Elmente in den Randzonen der Cluster sind nun durch die nachgeschaltete
Lokale Suche Prozedur mit der Nearest--Cluster Nachbarschaft in der Lage,
ihren Cluster zu wechseln. Da die Nearest--Cluster Nachbarschaft nur f"ur 
solche Elemente einen m"oglichen anderen Cluster zur"uckliefert, die
in ihrer unmittelbaren Umgebung ein Element aus einem anderen Cluster
besitzen, l"auft der 2. Schritt sehr effizient ab.

Das Verfahren Combined Algorithms ist also in gewisser Hinsicht die
Umkehrung von Simulated Annealing / Localopt mit der
Simultaneous--Centroid Nachbarschaft.
Denn bei der Vorgehensweise ist der Simultaneous--Algorithmus nicht der 
Hauptakteur, sondern nur der "`Nachoptimierer"'.
