\newcommand{\MAXCLUSTER}{15}


Die \verb-Werte fuer Anzeige- sind dabei auf 3 beschr"ankt (jeder
ben"otigt eine Dimension), w"ahrend zur Berechnung beliebig viele
Werte ausgew"ahlt werden k"onnen: Die weitere Auswahl
unter \verb-weitere Werte- erfolgt durch Klicken  mit der Maus
bei gleichzeitigem Dr"ucken der \verb-SHIFT-, bzw. \verb-Umschalt--Taste.





Dies ist allerdings nur dann m"oglich, wenn die so gew"ahlte Clusteranzahl
die H"ochstm"ogliche von \MAXCLUSTER nicht "uberschreitet.





Eine solche Clusterung ist auch immer dadurch zu erzielen, da"s man im
Werte-Auswahl-Men"u unter \verb-Werte fuer Berechnung- nur einen Wert, 
n"amlich den, der gerade vertikal verl"auft, angibt. Als Zielfunktion
(siehe \ref{Zielfunktionenkapitel}) sollte dabei m"oglicht \verb-SS-
gew"ahlt werden.


\item[Unterbereiche] Mit diesem Men"u ist es m"oglich solche Datens"atze
	von der Berechnung auszuschliessen, die unter einem betimmten
	Merkmal eine bestimmten Wert aufweisen.

	Beispielsweise hat ein Merkmal die Auspr"agungen \verb-Ja- und 
	\verb-Nein- und durch Marikieren von \verb-Nein- bleiben alle
	die Datens"atze unber"ucksichtigt, die dieses \verb-Nein-
	besitzen.

	In der Grafik werden weiterhin alle Punkte angezeigt, jedoch
	erscheinen diese "`passiven"' Punkte in wei"s.

	Dem Benutzer wird dadurch die M"oglichkeit gegeben, Abh"agigkeiten, 
	die nur in eine Untergruppe, z.B. Nichtrauchern, auftreten, 
	herauszufinden.

\subsubsection{Optionen}
\label{Zielfunktionenkapitel}
\begin{description}
\item[Alpha's und Suchalgorithmus] sind noch nicht endg"ultig und sollten
	bitte weder benutzt noch ver"andert werden. Ausnahme ist, da"s als
	Algorithmus auch \verb-Simulated Annealing mit Lokaler Suche-
	verwendet werden kann.
\item[Clusteranzahl] Hier kann man die gew"unschte Anzahl der Cluster
	eingeben, Voreinstellung ist 3, Maximalanzahl \MAXCLUSTER.
\item[Wichtung (Zielfunktionen)] Durch Anw"ahlen dieses Punktes (oder
	durch Dr"ucken der mittleren Maustaste, soweit vorhanden) erscheint
	ein Men"u, in dem die Wichtung der Zielfunktionen gegeneinander
	ge"andert werden kann.

	Die Bedeutung der einzelnen Zielfunktionen:
	\begin{itemize}
	\item {\bf SS}\\
		Es wird erst die Summe der euklidischen Abst"ande innerhalb
		der Cluster gebildet, und diese dann aufsummiert.
		Diese Zielfunktion liefert, wie auch die n"achste, in erster
		Linie Cluster m"oglicht gleicher Gr"o"se, die ann"ahernd
		konvexe Formen aufweisen. (Summe-Summe)
	\itemv {\bf SS\^2}\\
		Wie \verb-SS-, jedoch werden in der Clustern die 
		quadratischen Abst"ande summiert. (Summe-Quadrat)
	\item {\bf S 1/n*S}\\
		Hierbei wird die Summe in den Clustern noch durch die Anzahl
		der im Cluster vorhandenen Punkte dividiert - es erfolgt
		also eine Art Mittelung. Es zeigt sich, da"s hierbei
		F"alle auftauchen k"onnen, da"s einer oder mehrer Cluster
		nur noch aus einem einzigen Punkt bestehen!
		(Summe-1/n Summe)
	\item {\bf max S}\\
		Bei dieser Zielfunktion wird einfach das Maximun der Summen
		der Cluster betrachtet.  Auch dieses Verfahren terminiert
		i.a. nicht mit einem brauchbaren Ergebnis. (Max-Summe)
	\item {\bf S max}\\
		Wie sich nun leicht vermuten l"a"st, wird nun die Summe
		der Maxima in der Clusterm berechnet. D.h., da"s der
		Lokale-Suche-Algorithmus recht schnell terminiert, da ein
		Verschieben von Clusterpunkten, die nicht am 
		Zielfunktionswert beiteiligt sind, diesen nicht verbessern.
		W"ahlen von Simulated Anneling als Algorithmus schafft
		hier Abhilfe, jedoch tritt wieder das Ph"anomen auf, da"s 
		Cluster zu sogenannten trivialen Clustern, n"amlich aus
		einem Punkt bestehend, entarten k"onnen.
		(Summe-Max)
	\item {\bf max max}\\
		Bei dieser Wahl terminiert Lokale Suche am schnellsten mit
		einer unbrauchbaren L"osung. Diese Zielfunktion ist 
		nicht zu empfehlen. (Max-Max)
	\end{itemize}

	Die Zielfunktionen k"onnen beliebig kombiniert und gegeneinander
	gewichtet werden (dazu dienen die Zahlen - eine 0 bedeutet, 
	da"s dieses Zielfunktion nicht beachtet wird); der Vielfalt
	des Benutzers sind dabei kaum Grenzen gesetzt.

	Die Voreinstellung ist \verb-SS- und hat sich als recht brauchbar
	erwiesen.

\item[Nachbarschaft] Drei Nachbarschaften stehen hier zur Auswahl, wobei
	sich die voreingestellte -\verb-naechster Cluster-- als am 
	g"unstigsten erwiesen hat. N"aheres siehe hierzu in den Ausf"uhrungen
	der Diplomarbeit.
\item[Anfangspartition] Dieser Schalter hat die geringste Funktion und dient
	nur zu Testzwecken. Einstellung beliebig.
\end{description}



\subsubsection{Hilfe}

Wird noch ausgearbeitet.

