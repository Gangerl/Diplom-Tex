\chapter{Klassifikationsprobleme}
Am Anfang stellt sich die Frage, nach welchen Kriterien eine Menge
klassifiziert werden soll, und wieviele solcher Klassifizierungen
existieren.
Dieses Kapitel enth"alt dementsprechend eine Einf"uhrung in die Beschreibung von
Klassifikationsproblemen und deren Komplexit"at.
Am Ende erfolgt dann eine kurze Darstellung der verschiedenen
L"osungsstrategien.

\section{Allgemeines}
\begin{bezeichnung}
Sei $E = \{x_1,\dots,x_n\}$ die nichtleere, endliche Menge der zu 
gruppierenden $n$ Elemente und $\G$ eine {\em Gruppierung}
von $E$ in $k\ (2 \leq k \leq n-1)$
disjunkte Teilmengen oder {\em Cluster} $C_i$, deren Vereinigung
$E$ ist, also
\[ E =\bigcup_{i=1}^k C_i \qquad \mbox{und} \qquad 
        C_i \cap C_j = \emptyset \quad
        ( i,j = 1,\dots,k; \enspace i \neq j). \]
Die Anzahl der Elemente in den Clustern werde mit $n_l$ bezeichnet, 
es gilt also $n_l = |C_l|,\  n = \sum_{l=1}^k n_l$.
Statt Gruppierung sprechen wir auch von einer {\em Zerlegung}
oder {\em Partitionierung}.
Eine solche Partitionierung in $k$ Cluster nennt man dann auch
{\em $k$--Partitionierung} oder {\em $k$--Clustering}.
\end{bezeichnung}

\begin{definition}
Sei $f$ eine Funktion, die jeder $k$--Partitionierung eine reelle Zahl
$f(C_1,\dots,C_k)$ zuordnet. Ein {\em disjunktes Klassifikationsproblem} ist 
dann gegeben durch: Finde $C_1^*,\dots,C_k^*$ mit
\[ F(C_1^*,\dots,C_k^*) \ = \ \min\{F(C_1,\dots,C_k) \, | \, C_1,\dots,C_k
	\mbox{ ist eine $k$--Partitionierung} \}.\]
\end{definition}

Wir gehen also davon aus, da"s alle Elemente einem Cluster zugeordnet sind.
Man spricht dann von einer {\em exhaustiven} Gruppierung.

\begin{figure}[htbp]
\[\psfig{figure=b1-5}\]
\caption{Disjunkte, exhaustive Clusterung}
\end{figure}

\begin{figure}[htbp]
\[\psfig{figure=b1-6}\]
\caption{Nicht disjunkte, nicht exhaustive Clusterung}
\end{figure}

Zwischen den Elementen sind sogenannte {\em Distanzen} gegeben, durch 
die in der Regel die Funktion $f$ definiert wird. Die Distanz
zwischen den Elementen $x$ und $y$ bezeichnen wir mit $d(x,y)$. 
Ist keine Distanz zwischen $x$ und $y$ gegeben, so setzen wir
entweder $d(x,y) = 0$ oder $d(x,y) = \infty$, je nachdem was
angebracht ist.

Klassifikationsprobleme lassen sich h"aufig auch gut durch 
{\em ungerichtete Graphen} darstellen. Ein ungerichteter Graph $G=(V,E)$ 
besteht aus einer Knotenmenge $V$ und einer Kantenmenge $E \subseteq
P_2(V)$ = 2--elementige Teilmengen von $V$.

\begin{figure}[htbp]
\begin{center}
\input b1-3.pic
\end{center}
\caption{ungerichteter, gewichteter Graph}
\end{figure}

Die Knoten entsprechen dabei den Elementen, die Gewichte der 
Kanten symbolisieren die Distanzen zwischen ihnen.

\begin{bezeichnung}
F"ur eine genaue Betrachtung f"uhren wir folgende Bezeichnungen ein:
\begin{quote}
	Ein beliebiges Clustering--Problem mit $k$ Clustern und Zielfunktion $F$
	bezeichnen wir mit $k|\cdot|F$; befinden sich die Elemente im 
	$m$--di\-men\-sio\-na\-len euklidischen Raum, so schreiben wir
	$k|m$--dim.--Euklid$|F$; oder der Einfachheit halber $k|m|F$.
\end{quote}
\end{bezeichnung}

\section{Zielfunktionen}
\label{zielfunktionen}
\label{zielfunktionenkapitel}
Problemabh"angig kann man sich mehrere solcher Funktionen "uberlegen,
wir werden im Rahmen dieser Arbeit die folgenden sechs betrachten 
(vgl. auch Brucker [1978]):

\begin{eqnarray*}
F_1(C_1,\dots,C_k) & = &  \sum\limits_{l=1}^k \sum\limits_{i,j \in C_l 
	\atop i \neq j} d(x_i,x_j) \  \qquad =:  \ \sum \sum d_{ij} \\
F_2(C_1,\dots,C_k) & = &  \sum\limits_{l=1}^k \sum\limits_{i,j \in C_l 
	\atop i \neq j} (d(x_i,x_j))^2\   \quad =:\ \sum \sum d_{ij}^2 \\
F_3(C_1,\dots,C_k) & = &  \sum\limits_{l=1}^k \frac{1}{|C_l|}\sum\limits_{i,j 
	\in C_l \atop i \neq j} d(x_i,x_j)\ \quad  =:\ \sum \frac 1n \sum d_{ij} \\
F_4(C_1,\dots,C_k) & = &  \max\limits_{l=1}^k \Big\{\sum\limits_{i,j \in C_l 
	\atop i \neq j} d(x_i,x_j)\Big\} \ \quad =:  \ \max \sum d_{ij} \\
F_5(C_1,\dots,C_k) & = &  \sum\limits_{l=1}^k \max\limits_{i,j \in C_l 
	\atop i \neq j} \big\{d(x_i,x_j)\big\} \qquad \ =:  \ \sum \max d_{ij} \\
F_6(C_1,\dots,C_k) & = & \max\limits_{l=1}^k \Big\{\max\limits_{i,j \in C_l 
	\atop i \neq j} \big\{d(x_i,x_j)\big\}\Big\}  \ =: \  \max \max d_{ij} 
\end{eqnarray*}

Auf die Abstandsfunktion $d(x,y)$ wird im n"achsten Abschnitt
\ref{distanzfunktion} noch n"aher eingegangen;
an dieser Stelle sei nur erw"ahnt, da"s wir uns haupts"achlich im
$m$--dimensionalen euklidischen Raum bewegen werden.
Es gilt also $|| x - y || = (\sum_{i=1}^m |y_i - x_i |^2)^{\frac 12}$.

Viele Autoren verwenden die von Bock [1974] {\em Varianzkriterium} 
genannte Zielfunktion
\begin{equation}
	Z_1(C_1,\dots,C_k) \ =  \ \sum_{l=1}^k \sum_{i \in C_l}
		|| x_i - \bar x_l ||^2,
\end{equation}
wobei 
$\bar x_l := \frac 1 {n_l} \sum_{i \in C_l} x_i$.
der {\em Gruppencentroid} oder {\em Gruppenschwerpunkt}
des Clusters $C_l$ ist. 
Beschr"anken wir uns mit unserer Abstandsfunktion $d(x,y)$ auf den
euklidischen Fall, so k"onnen wir zeigen, da"s das Varianzkriterium
"aquivalent zu der Zielfunktion $F_3$ ist:

Sei dazu
\[ \bar x \ = \ \frac 1 n \sum _{i=1}^n x_i \]
der {\em Schwerpunktvektor}, f"ur den aufgrund seiner Definition
\begin{equation}
	\label{ximinusxquer}
	\sum_{i=1}^n (x_i - \bar x) \ =  \ 0
\end{equation}
gilt.

\begin{lemma}
	Ist $y \in \R^m$ ein beliebiger reller Vektor, so gilt der
	sogenannte {\em Verschiebungssatz}
	\begin{equation}
		\label{verschiebungssatz}
		\sum_{i=1}^n || x_i - y||^2 \ = \ \sum_{i=1}^n ||x_i - \bar x||^2 +
		n || \bar x - y ||^2 .
	\end{equation}
\end{lemma}
{\bf Beweis:}
\vspace*{-3mm}
\begin{eqnarray*}
	x_i - y & = & (x_i -\bar x) + (\bar x - y) \\
	\Rightarrow \quad || x_i - y ||^2 & = & || x_i - \bar x ||^2 + 2(x_i -
	\bar x)^T (\bar x - y) + || \bar x - y ||^2 \\
	\Rightarrow \quad \sum_{i=1}^n || x_i - y ||^2 & = & \sum_{i=1}^n 
	|| x_i - \bar x ||^2 + n ||\bar x - y ||^2 + 2  \sum_{i=1}^n
	(\bar x - y)^T(x_i - \bar x).
\end{eqnarray*}
Der letzte Term auf der rechten Seite verschwindet wegen (\ref{ximinusxquer}), 
und damit ist der Verschiebungssatz gezeigt.$\quad\Box$

Setzt man in (\ref{verschiebungssatz}) $y = x_j$ und summiert "uber
$j = 1,\dots,n$, so erh"alt man
\begin{eqnarray*}
	\sum_{j=1}^n \sum_{i=1}^n ||x_i - x_j ||^2 & = &
		\sum_{j=1}^n \left( \sum_{i=1}^n ||x_i - \bar x ||^2 
		+ n ||\bar x - x_j||^2\right) \\
	& = & n\sum_{i=1}^n || x_i - \bar x ||^2 + n \sum_{j=1}^n ||\bar x
		- x_j ||^2\\
	\Leftrightarrow \quad \sum_{i=1}^n || x_i - \bar x ||^2 & = &
		\frac 1 n \sum_{j=1}^n \sum_{i=1}^n || x_i - x_j||^2 - 
		\sum_{j=1}^n ||\bar x - x_j ||^2 
\end{eqnarray*}
und schlie"slich
\[ \sum_{i=1}^n ||x_i - \bar x||^2 \ = \ \frac 1{2n} \sum_{i=1}^n \sum_{j=1}^n
	||x_i - x_j ||^2. \]
Schreibt man dies f"ur einen durch $C_l$ charakterisierten Cluster, 
so bekommt man
\[ \sum_{i \in C_l} || x_i - \bar x_l ||^2 \ = \ 
	\frac 1{2n_l} \sum_{i,j \in C_l} || x_i - x_j ||^2.\]
Hieran erkennt man, da"s die Summe der quadratischen Abweichungen der
Clustermitglieder von ihrem Schwerpunkt allein durch die Abst"ande
der Clustermitglieder untereinander beschrieben werden kann.
Bis auf den konstanten Faktor $\frac 1 2$ und das Quadrat an den 
Distanzen stimmen die Funktionen $Z_1$ und $F_3$ also "uberein. 
W"ahlt man $d(x_i,x_j) = \frac 12 || x_i - x_j ||^2$, so
hat man eine vollst"andige "Ubereinstimmung.

{\bf Beispiel} (siehe Abbildung \ref{varianzbild}):
Gegeben seien ein Cluster $C_1$ mit den Punkten $P_1(1;1), P_2(3;2),
$ und $P_3(2;3)$, sowie ein Cluster $C_2$ mit dem Punkt $P_4(5;5)$.
Dann betr"agt
\begin{eqnarray*}
\bar x_1 & = & \left(\frac 13 (1+3+2); \frac 13 (1+2+3) \right) \ = \
	(2,2),\\
\bar x_2 & = & (5,5),\\
Z_1(C_1) & = & 1^2 + 1^2 + 1^2+0^2+0^2+1^2 \ = \ 4,\\
Z_1(C_2) & = & 0,\\
Z_1(C_1,C_2) & = & 4.
\end{eqnarray*}

\begin{figure}[htbp]
\[\input b1-7.pstex_t\]
\caption{Darstellung zum Varianzkriterium \label{varianzbild}}
\end{figure}

Mit diesen Zielfunktionen werden nun bestimmte Erwartungen verkn"upft:
Zum einen sollen die Cluster nicht "`entarten"', das hei"st sie
sollen zusammenh"angend und m"oglichst konvex\footnote{Eine Menge
$M \subset \R^n$ hei"st {\em konvex}, wenn sie mit jedem Punktepaar 
$x,y \in M$ die ganze Verbindungsstrecke zwischen $x$ und $y$ enh"alt.
Die {\em konvexe H"ulle} $\bar N$ einer Menge $N \subset \R^n$  ist 
die kleinste konvexe Menge, die $N$ enth"alt.}
 sein (geometrisch 
betrachtet), und sie sollen die tats"achliche Struktur der Objekte
gut widerspiegeln. 
Konvexit"at deshalb, da es sonst nicht angebracht scheint, 
"uberhaupt eine Clusterung vorzunehmen.

Von $\sum\sum$ und $\sum\sum^2$ erwarten wir, da"s die Anzahlen der Elemente 
in den
Clustern bei einer gleichm"a"sig verteilten Menge etwa gleich sind.
Da n"amlich jeder Cluster eine Clique bildet (Clique = vollst"andig
vernetzter Graph), und die Kantenanzahl in einer Clique quadratisch
mit der Knotenzahl w"achst, existieren somit insgesamt mehr Kanten, wenn
ein Cluster "uberproportional viele Elemente enth"alt, als wenn die
Elementanzahlen in den Clustern etwa gleich sind. Dadurch w"urden
also mehr Kantengewichte aufsummiert.

\begin{figure}[htbp]
\begin{center}
\input b1-4.pic
\end{center}
\caption{M"ogliche 2--Partitionierungen eines Problems}
\end{figure}

Bei $\sum\sum^2$ machen sich gro"se $d_{ij}$--Werte st"arker bemerkbar als bei
$\sum\sum$, das hei"st die Cluster werden eher "`Kugelgestalt"' annehmen.

Das Streben nach gleichen Elementanzahlen in den Clustern ist bei 
$\sum\frac 1n \sum$ nicht gegeben, da hier diese Anzahl mit einflie"st, 
und sowohl gro"se als auch kleine Cluster entsprechend ihrer
Gr"o"se gewichtet werden.

Im Gegensatz hierzu ist $\max\sum$ wieder bem"uht die Summen in den Clustern
einem Wert $S$ anzugleichen; im optimalen Fall gilt 
$\sum_{i,j \in C_l \atop i\neq j } d_{ij} = S \ \forall\, l = 1,\dots,k$.
Das hei"st geometrisch gesehen, da"s Cluster, die r"aumlich sehr gro"s
sind, nur wenige Elemente; r"aumlich kleine dagegen viele Elemente
enthalten werden.

Bei $\sum\max$ spielen spielen die Anzahlen der Elemente in den jeweiligen
Clustern keine so gro"se Rolle mehr, wie auch bei $\max\max$. 
Anders als bei den ersten 4 Zielfunktionen k"onne wir also keine
Aussage dar"uber treffen, wieviele Elemente in den einzelnen Clustern
sein werden.
Jedoch lassen sich die Maximierungen dahingehend interpretieren, 
da"s wir bei $\max\max$ r"aumlich  gleich gro"se Cluster erwarten k"onnen.

Welche Zielfunktion jetzt f"ur eine Datenmenge und Clusteranzahl die
geeignetste ist, wird sich zeigen m"ussen. 
Vorab l"a"st sich h"ochstens sagen, da"s Zielfunktionen, 
die Cluster mit gleichen Elementanzahlen hervorbringen, 
nicht geeignet sein werden solche Partitionierungen vorzunehmen, 
bei denen sehr unterschiedlich gro"se Cluster eine gute 
L"osung darstellen.

\section{Distanzfunktion}
\label{distanzfunktion}
\begin{definition}
Eine Abbildung $d:E\times E \to [0;\infty)$ hei"st eine 
{\em Abstandsfunktion}, wenn gilt
\begin{eqnarray*}
	d(x,y)  =  d(y,x)  \geq  0 & \qquad & \mbox{(Symmetrie)}\\
	d(x,y) = 0 \ \Rightarrow  \ x = y &       & \mbox{(Eindeutigkeit)}\\
	d(x,z) + d(z,y)  \geq  d(x,y) &       & \mbox{(Dreiecksungleichung)}
\end{eqnarray*}
\end{definition}
In praktischen Anwendungen wird die Eindeutigkeit oft vernachl"assigt.

Die {\em Minkowski}--Metriken sind bekannte Beispiele f"ur 
Distanzfunktionen, f"ur die f"ur $p \geq 1$ und 2 Punkte $x_i,x_j$ gilt
\[ L_p (x_i,x_j) \ = \ \left(\sum_{k=1}^m |x_{jk} - x_{ik} |^p
	\right) ^{\frac 1 p}. \]
Diese Familie enth"alt die $L_1$-- oder {\em Manhatten}--Metrik, die
$L_2$ oder euklidische Metrik, und in ihrem Grenz"ubergang
$p \to \infty$ $L_\infty$ oder die Tschebyscheff--Metrik, f"ur die gilt
\[ L_\infty(x_i,x_j) \ = \ \max \{ | x_{ik} - x_{jk} | : 1 \leq k \leq m \}.\]

Die geometrische Anschauung legt nahe, den Abstand zweier Elemente
$e_i, e_j$ durch den euklidischen Abstand der sie repr"asentierenden
Punkte $x_i, x_j$ zu messen:
\[ d(x_i, x_j) \ := \ || x_j - x_i || \ = \ 
	\sqrt{ \sum_{k=1}^m (x_{jk} - x_{ik})^2} \]

Die Eigenschaft dieser Distanzfunktion l"a"st sich so beschreiben:
Die Distanzen $d(x_i, x_j)$ sind translationsinvariant sowie
invariant bez"uglich orthogonaler linearer Transformationen 
(Drehungen, Spiegelungen) der Vektoren $x_1,\dots,x_n$; sie 
"andern sich also nicht bei Abbildungen der Art
\[ x_i \to A x_i + b \quad i = 1,\dots,n, \]
wobei $b \in \R^m$ ein beliebiger, fester Vektor und $A$ eine
orthogonale $m \times m$--Matrix ist.
Die euklidische Distanz ist allerdings nicht skaleninvariant;
daf"ur m"u"sten die Daten normiert sein.

Einige Autoren verwenden auch die {\em Mahalanobis}--Distanz
\begin{eqnarray*}
	d^2 (x_i,x_j) & = & (x_i -x_j) \Sigma^{-1} (x_i - x_j)^T \\
				  & = & \sum_{k=1}^m \sum_{l=1}^m (x_{ik} - x_{jk})
					\sigma_{kl} (x_{il} - x_{jl}).
\end{eqnarray*}
Dabei ist $\Sigma^{-1} = (\sigma^\ast_{kl})$ die Inverse der empirischen
$m\times m$--Kovarianzmatrix $\Sigma = (\sigma_{kl})$ von
$x_1,\dots,x_n$ mit den Elementen
\[ \sigma_{kl} \ := \ \frac 1{n-1} \sum _{i=1}^n (x_{ik} - \bar x _{.k})
	(x_{il} - \bar x _{.l})\qquad k,l = 1,\dots,m. \]
Hierbei gilt
\[ \bar x_{.i} \ := \ \frac 1 n \sum_{k=1}^n x_{ki}. \]

Die Eigenschaften der Mahalanobis--Distanz k"onnen folgenderma"sen 
charakterisiert werden:
\begin{enumerate}
\item $d_{ij}$ h"angt von allen $n$ Vektoren $x_1,\dots,x_n$ ab. Es gilt
\[ 0 \leq d_{ij}^2 \leq 2n \quad \mbox{ und } \quad 
	\sum_{i=1}^n\sum_{j=1}^n d_{ij}^2 = 2n^2m.\]
\item $d_{ij}$ ist invariant bez"uglich aller Parallelverschiebungen
und aller linearen, nicht singul"aren Transformationen der $x_1,\dots,x_n$:
\[ x_i \to \tilde x_i := A x_i + b \quad i =1,\dots,n,\]
wobei $b \in \R^m$ ein beliebiger, fester Vektor und $A$ eine reelle
$m\times m$--Matrix mit $\det(A) \neq 0$ ist. Bei dieser
Transformation geht n"amlich
\begin{eqnarray*}
	\tilde x \mbox{ in } \tilde{\tilde x} & := & A \tilde x + b \ \mbox{ und}\\
	\Sigma   \mbox{ in } \tilde\Sigma     & := & A \Sigma A^T \quad \mbox{mit }
		\tilde\Sigma^{-1} = {C^T}^{-1} \cdot \Sigma^{-1} \cdot C^{-1}
\end{eqnarray*}
"uber, so da"s sich f"ur die Distanz $\tilde d_{ij}^2$ der transformierten
Gr"o"sen gerade
\begin{eqnarray*}
	\tilde d_{ij}^2 & := & (\tilde x_i -\tilde x_j)^T\tilde\Sigma^{-1}
		(\tilde x_i - \tilde x_j) \\
	& = & (x_i -x_j)^TC^T{C^T}^{-1}\cdot\Sigma^{-1}\cdot C^{-1}C(x_i-x_j) \\
	& = & (x_i -x_j)^T \Sigma^{-1}(x_i-x_j)\\
	& = & d^2_{ij}
\end{eqnarray*}
ergibt.
\end{enumerate}
Die Mahalanobis--Distanz besitzt somit weitergehende Invarianzeigenschaften
als die euklidische Distanz, bei der die Matrix $C$ orthogonal sein mu"s;
insbesondere ist hier die Skaleninvarianz gesichert.

Weitere Eigenschaften dieser Distanzfunktion findet man bei Mahanalobis
[1936]. 
Der Vorteil der Skaleninvarianz der Mahalanobis--Distanz geht bei uns
dadurch verloren, da"s wir die zugrundeliegenden Daten sowieso normieren
m"ussen, um eine Visualisierung des Rechenprozesses zu erm"oglichen.
Wir werden also die euklidische Metrik als Distanzfunktion verwenden.

\section{Komplexit\"at}
\begin{satz}
Die Anzahl $S(n, k)$ der bei $k$ Clustern m"oglichen disjunkten Zerlegungen
$\G$ von $E$ l"a"st sich "uber die Formel
\begin{equation}
        \label{stirling}
        S(n, k) \; = \; \frac{1}{k!} \sum_{i=0}^k (-1)^i {k \choose i} (k-i)^n 
\end{equation}
berechnen, bzw. im Anwendungsfall einfacher rekursiv "uber
\begin{equation}
        \label{stirlingrekursiv}
        S(n+1, k) \; = \; S(n, k-1) + k \cdot S(n, k) \quad\mbox{ mit }
        \begin{array}{l}
                 %S(n, 1)  =  S(n, n) = 1, \\
                 %S(n, k)  =  0 \ \mbox{f"ur } n < k.
				 S(0,0) = 1,\\
				 S(n,0) = 0 \mbox{ f"ur } n > 0,\\
				 S(n,1) = 1 \mbox{ f"ur } n > 0,\\
				 S(n,k) = 0 \mbox{ f"ur } n < k.
         \end{array}
\end{equation}
\end{satz}
{\bf Beweis} der Rekursionsformel:
Man kann die Zerlegungen von $\{x_0,x_1,\dots,x_n\}$ aufteilen in die
Partitionen, die den Cluster $\{x_0\}$ enthalten 
(die Menge $\{x_1,\dots,x_n\}$ mu"s dann noch in $k-1$ Cluster
aufgeteilt werden), und in die Partitionen, bei denen das Element $x_0$
zu einem der $k$ Cluster einer Zerlegung von $\{x_1,\dots,x_n\}$ 
geschlagen wird.

Der Beweis von (\ref{stirling}) erfolgt durch Induktion:
Nat"urlich kann (\ref{stirling}) geschrieben werden als
\begin{equation}
	\label{stirling2}
	S(n,k) \; = \; \frac{1}{k!} \sum_{i=0}^k (-1)^{k-i} {k \choose i} i^n .
\end{equation}
F"ur $n=0$ lautet die rechte Seite dann
\[
	\frac{1}{k!} \sum_{i=0}^k (-1)^{k-i} {k \choose i} \; = \;
	\frac{1}{k!} \sum_{i=0}^k { k \choose i} 1^i (-1)^{k-i} \; = \;
	\frac{(1-1)^k}{k!},
\]
was f"ur $k=0$ den Wert $1$ und sonst den Wert $0$ ergibt. F"ur $n>0=k$ ist
die rechte Seite von (\ref{stirling2}) gleich $0 = S(n,k)$. 
Angenommen (\ref{stirling2}) gilt f"ur $n$ und alle $k=0,1,\dots$;
dann liefert (\ref{stirlingrekursiv}) f"ur $k \geq 1$
\begin{eqnarray*}
	S(n+1,k) & = & k S(n,k) + S(n,k-1) \\
		 & = & \frac {k}{k!} \sum_{i=0}^k (-1)^{k-i} { k \choose i} i^n
		     + \frac {1}{(k-1)!} \sum_{i=0}^{k-1} (-1)^{k-1-i}
		       {k-1 \choose i} i^n \\
		 & = & \frac {1}{(k-1)!} \sum_{i=0}^k (-1)^{k-i} \,i^n
		       \left[{k \choose i} - { k-1 \choose i} \right] \\
		 & = & \frac {1}{(k-1)!} \sum_{i=0}^k (-1)^{k-i} \,i^n
		       {k-1 \choose i-1}  \\
		 & = & \frac {1}{k!} \sum_{i=0}^k (-1)^{k-i} \,i^{n+1}
		       \,\frac ki {k-1 \choose i-1}  \\
		 & = & \frac {1}{k!} \sum_{i=0}^k (-1)^{k-i} 
		       {k \choose i} i^{n+1}. \quad\quad\Box 
\end{eqnarray*}
\vspace*{0.5cm}

Die $S(n, k)$ werden auch {\em Stirling'sche Zahlen zweiter Art} genannt
und besitzen etwa f"ur $n = \{1,2,\dots,10,15,20,25,50,100\}$ und 
$k = \{1,2,3,4,5,10\}$ die Werte aus Tabelle \ref{tabsterling}.

\begin{table}
\caption{Stirling'sche Zahlen zweiter Art \label{tabsterling}}
\begin{center}
\begin{tabular}{|r||rrrrrrr|}\hline
	%\multicolumn{9}{|c|}{\rule[-3mm]{0mm}{8mm} 
		%Stirling'sche Zahlen zweiter Art }\\
	%\hline
	$\displaystyle \bslashfrac{k}{n}$ & 0 & 1 & 2 & 3 & 4 & 5 & 10 \\
	\hline\hline
	0    & 1 & 0 & 0 & 0 & 0 & 0 & 0  \\
	1    & 0 & 1 & 0 & 0 & 0 & 0 & 0  \\
	2    & 0 & 1 & 1 & 0 & 0 & 0 & 0  \\
	3    & 0 & 1 & 3 & 1 & 0 & 0 & 0  \\
	4    & 0 & 1 & 7 & 6 & 1 & 0 & 0  \\
	5    & 0 & 1 & 15 & 25 & 10 & 1  & 0  \\
	6    & 0 & 1 & 31 & 90 & 65 & 15 & 0  \\
	7    & 0 & 1 & 63 & 301 & 350 & 140 & 0  \\
	8    & 0 & 1 & 127 & 966 & 1701 & 1050 & 0  \\
	9    & 0 & 1 & 255 & 3025 & 7770 & 6951 & 0  \\
	10   & 0 & 1 & 511 & 9330 & 34105 & 42525 & 1  \\
	15   & 0 & 1 & 16383 & 2.3$\cdot 10^{6}$ & 4.2$\cdot 10^{7}$ & 2.1$\cdot 10^{12}$ & 1.2$\cdot 10^{7}$  \\
	20   & 0 & 1 &  524287 & 5.8$\cdot 10^{8}$ & 4.5$\cdot 10^{10}$ & 7.5$\cdot 10^{11}$ & 5.9$\cdot 10^{12}$  \\
	25   & 0 & 1 & 1.7$\cdot 10^{7}$ & 1.4$\cdot 10^{11}$ & 4.7$\cdot 10^{13}$ & 2.4$\cdot 10^{15}$ & 1.2$\cdot 10^{18}$  \\
	50   & 0 & 1 & 5.6$\cdot 10^{14}$ & 1.2$\cdot 10^{25}$ & 5.3$\cdot 10^{28}$ & 7.4$\cdot 10^{32}$ & 2.6$\cdot 10^{43}$  \\
	100  & 0 & 1 & 6.3$\cdot 10^{31}$ & 8.6$\cdot 10^{46}$ & 6.7$\cdot 10^{58}$ & 6.6$\cdot 10^{67}$ & 2.8$\cdot 10^{93}$  \\
	\hline
\end{tabular}\\
\end{center}
\end{table}

Ebenso einfach einzusehen ist das 
\begin{lemma}
Es gilt
\begin{equation}
	\label{stirlingrek}
	S(n+1,k) \; = \; \sum_{i=0}^n {n \choose i} S(i,k-1) \quad
		(n=0,1,\dots;\ k=1,2,\dots).
\end{equation}
\end{lemma}
{\bf Beweis:}
Dies beruht auf folgender Zerlegung von $\{x_0,x_1,\dots,x_n\}$: 
1) Festlegung
des $x_0$ enthaltenden Cluster mit $n+1-i$ Elementen durch Fixierung einer
Teilmenge von $\{x_1,\dots,x_n\}$ mit $n-i$ Elementen
$\Big( {n \choose n-i} = {n \choose i }$ M"oglichkeiten $\Big)$;
2) Zerlegung der Restmenge von $i$ Elementen in $k-1$ Cluster.$\quad\Box$
\vspace*{0.5cm}


Wird die Anzahl $k$ der Cluster nicht vorherbestimmt, so kann man die 
Anzahl $B_n$ aller m"oglichen Partitionen rekursiv ermitteln, indem
man die Formel
\[ B_n = \sum_{i=1}^n S(n, i)\]
verwendet. Es gilt nun der 

\begin{satz}
F"ur die Anzahl $B_{n}$ aller m"oglichen Partitionierungen von $n$ 
Elementen gilt
\[ B_{n+1} = \sum_{j=0}^n {n \choose j} B_j \quad \mbox{mit } 
	B_0 = 1,\enspace B_1 = 1. \]
\end{satz}
{\bf Beweis: }
\begin{eqnarray*}
	B_{n+1} & = & \sum_{i=1}^{n+1} S(n+1,i) \\
		& {(\ref{stirlingrek}) \atop =} & \sum_{i=1}^{n+1}
			\sum_{j=0}^n {n \choose j} S(j,i-1) \\
		& = & \sum_{j=0}^n {n \choose j} \sum_{i=1}^{n+1} S(j,i-1) \\
		& = & \sum_{j=0}^n {n \choose j} \sum_{i=0}^n S(j,i). \\
%		& = & \sum_{j=0}^n {n \choose j} B_j,
\end{eqnarray*}
F"ur $n=0$ ist der Wert der zweiten Summe 1,
% und f"ur $n > 0$ gerade
%$B_j$, da $B_j = \sum_{i=1}^j S(j,i) = \sum_{i=1}^n S(n,i)$ f"ur $n>j$.
somit gilt $B_1 = { 0 \choose 0} \cdot 1 = 1$. 
F"ur $n > 0$ gilt $\sum_{i=0}^n S(j,i) = \sum_{i=1}^n S(j,i) = B_j$,
da $n > j$.
$\quad\Box$
\vspace*{0.5cm}

%da $B_n = \sum_{i=0}^n S(n,i) = \sum_{i=0}^\infty S(n,i)$.
Diese Zahlen werden {\em Bell'sche Zahlen} oder {\em Exponentialzahlen}
genannt.

\begin{table}[htbp]
\caption{Bell'sche Zahlen}
\begin{center}
\begin{tabular}{|r|r||r|r|}\hline
	%\multicolumn{4}{|c|}{\rule[-3mm]{0mm}{8mm} Bell'sche Zahlen }\\
	%\hline
	n & $B(n)\enspace$ & n &  $B(n)\enspace$  \\
	\hline
	1 & 1 & 15 & 1.4$\cdot 10^{7}$ \\
	2 & 2 & 20 & 5.2$\cdot 10^{13}$ \\
	3 & 5 & 25 & 4.6$\cdot 10^{17}$ \\
	4 & 15 & 30 & 8.5$\cdot 10^{23}$ \\
	5 & 52 & 35 & 2.8$\cdot 10^{29}$ \\
	6 & 203 & 40 & 1.6$\cdot 10^{35}$ \\
	7 & 877 & 45 & 1.4$\cdot 10^{41}$ \\
	8 & 4140 & 50 & 1.8$\cdot 10^{47}$ \\
	9 & 21147 & 60 & 9.8$\cdot 10^{59}$ \\
	10 & 115975 & 70 & 1.8$\cdot 10^{73}$ \\
	\hline
\end{tabular}\\
\end{center}
\end{table}

\section{\NP--Vollst"andigkeit}
\label{npvollstaendigkeit}
Anhand obiger Formeln f"ur die $S(n,k)$ l"a"st sich erahnen, da"s es sich
bei Klassifikation im allgemeinen um ein \NP--schwieriges Problem handelt;
nur einige Spezialf"alle sind polynomial l"osbar.

Bisher haben wir die Probleme formuliert als:
\begin{quote}
	"`Bestimme eine Partitionierung derart, da"s der Zielfunktionswert
	  minimal wird."'
\end{quote}
In der theoretischen Untersuchung spielen Entscheidungsprobleme eine
wichtige Rolle. Ein Entscheidungsproblem besteht aus einer
Fragestellung, die entweder mit "`ja"' oder "`nein"' beantwortet werden
kann. Wenn wir obige Problemstellung als Entscheidungsproblem
formulieren erhalten wir:
\begin{quote}
	"`Gegeben eine Zahl $k \in \N$. Gibt es eine Partitionierung
	  mit Zielfunktionswert $\leq k$ ?"'
\end{quote}
Ein Klassifikationsproblem $P$ ist \NP--schwierig, falls das zugeh"orige
Entscheidungsproblem $P(k)$ \NP--vollst"andig ist.
F"ur eine genauere Einf"uhrung in die Theorie der \NP--Vollst"andigkeit 
siehe man 
das Standardwerk von Garey \& Johnson [1979]; hier sei nur erw"ahnt, da"s ein
Problem $P'$ {\em reduzierbar} auf eine Problem $P$ ist ($P' \alphared P$), 
wenn eine polynomial berechenbare Funktion $g$ existiert, die jeden Input
von $P'$ in einen Input von $P$ so "uberf"uhrt, da"s gilt:
\begin{quote}
	$x$ ist "`Ja"'--Input von $P'$ $\iff$ $g(x)$ ist "`Ja"'--Input von $P$.
\end{quote}

Um zu zeigen, da"s ein Entscheidungsproblem $P$ \NP--vollst"andig ist, 
mu"s gezeigt werden, da"s
\begin{enumerate}
\item $P$ in der Klasse \NP\ liegt, das hei"st 
	$P$ in polynomialer Zeit durch eine nichtdeterministische Turing
	Maschine gel"ost werden kann.
	%es existiert ein Zertifikat $y$
	%derart, da"s zu jedem "`Ja"'--Input $x$ ein polynomialer Algorithmus
	%existiert, der best"atigt, da"s $y$ ein g"ultiges Zertifikat ist, 
	%wobei $|y|$ durch ein Polynom in $|x|$ beschr"ankt ist.
\item Es existiert ein \NP--vollst"andiges Problem $P'$ mit $P' 
	\alphared P$.
\end{enumerate}

Die Klassifikationsprobleme, die wir hier betrachten, liegen alle in der
Klasse \NP, denn sie k"onnen durch Backtracking auf
einer nichtdeterministischen Turing Maschine in  polynomialer Zeit
gel"ost werden.

\begin{satz}
	F"ur alle $k \geq 2, \ \beta =$"`$\cdot$"' oder 
	$\beta=$"`$m$--dim.--Euklid$"'$ ($m$ beliebig) und Zielfunktion $f$ gilt
		\[  k | \beta |f \ \alphared\  k+1|\beta|f. \]
\end{satz}
{\bf Beweis:}
Angenommen, der Graph $G=(V,E)$ entspreche dem Problem $k|\cdot|f$ mit
$n$ Knoten, und $S$ sei eine obere Schranke f"ur $f$. Dann konstruieren
wir das Problem $k+1 | \cdot | f$ mit $V^* = V \cup \{v^*\}$, wobei
$v^* \not\in V,\  E^* = E \cup \{\{i,v^*\} \, | \, i \in V\}$ und 
$d_{iv^*} > nS$. Die anderen $d_{ij}$--Werte bleiben erhalten.
Eine optimale L"osung des erweiterten Problems mu"s einen 1--elementigen
Cluster $\{v^*\}$ enthalten, und die "ubrigen Cluster stellen eine
optimale L"osung des Originalproblems dar.$\quad\Box$
\vspace*{0.5cm}

\subsection{\NP--vollst"andige Probleme}
Das {\em Partitionsproblem} (PART) wurde von Garey \& Johnson [1979] als
\NP--vollst"andig klassifiziert und ist wie folgt definiert: Gegeben ist eine
Menge von positiven ganzen Zahlen $x_1,\dots,x_n$. Gesucht ist eine
Teilmenge $J \subset \{1,\dots,n\}$ mit der Eigenschaft
\[ \sum_{i \in J} x_i \ = \ b \ := \ \frac12 \sum_{i=1}^n x_i.\]

%Wir betrachten nun das Problem $2|\cdot|\max\sum d_{ij}$. Dies liegt in
%\NP, denn:
%Gibt es eine L"osung mit Zielfunktionswert $Z \leq b \ (b \in \N)$?
%F"ur jeden "`Ja"'--Input $x$ mit $|x|$ = $O(n)$ existiert ein Zertifikat
%$y = \{\sum_{i,j \in C_1 \atop i \neq j} d_{ij},
	%\sum_{i,j \in C_2 \atop i \neq j} d_{ij} \}$
%mit $|y|$ = $O(n)$ und in Zeit $O(n)$ kann dies "uberpr"uft werden.

\begin{satz}
\label{satz19}
	\begin{quote}
		PART\ $\alphared \ 2 |\cdot|\max\sum d_{ij}$.
	\end{quote}
\end{satz}
{\bf Beweis:}
Das Partitionsproblem sei gegeben durch die Werte $x_1,\dots,x_n$ 
und $b=\frac12 \sum_{i=1}^n x_i$. Angenommen, das Partitionsproblem habe
als L"osung die Menge $I$. Dann gilt f"ur das Klassifikationsproblem mit
der Knotenmenge
\[ V = \{1,\dots,n\} \cup \{ i' \, | \, i = 1,\dots,n \} \cup
	\{ i^* \, | \, i = 1,\dots,n\}, \]
der Kantenmenge
\[ 	E = \{ \{i,i'\} \, | \, i = 1,\dots,n\} \cup \{\{i,i^*\}\,|\,i=1,\dots,n\}
	\cup \{\{i',i^*\} \, | \, i = 1,\dots,n\} \]
und den Distanzen
\[ d_{ii^*} = d_{i'i^*} = 2b, \quad d_{ii'} = x_i \quad\forall\, 
	i=1,\dots,n\,,\]
da"s der Zielfunktionswert nicht gr"o"ser als $b$ ist.
Denn die Partition definiert durch
$C_1 = \{ i \, | \, i \in I\} \cup \{i' \, | \, i \in I \}
\cup \{i^* \, | \, i \not\in I\}$ hat den Zielfunktionswert $b$.
Betrachte dazu Abbildung \ref{beweis19} 
(es sei O.B.d.A $I = \{1,\dots,l\},\ l<n$).

%%%%%%%%%%%%%%%\input b1-1.pic
\begin{figure}[htbp]
%\[\input b1-1neu.pstex_t \]
\caption{\label{beweis19}Zum Beweis des Satzes}
\end{figure}

Hat andererseits das Klassifikationsproblem die  L"osung $C_1,C_2$ mit
$C_1 \cup C_2 = \{1,\dots,n\}$ und Zielfunktionswert $\leq b$, dann mu"s
$i^*$ in einem anderen Cluster als $i$ und $i'$ liegen. 
Somit gilt $i \in C_1$ genau dann, wenn $i' \in C_1$. Also gibt es
eine Menge $I \subseteq \{1,\dots,n\}$ mit
\[ \sum_{ i,j \in C_1 \atop i \neq j} d_{ij} = \sum_{i \in I} x_i \leq b
	\ \mbox{ und } \ 
	\sum_{i,j \in C_2 \atop i \neq j} d_{ij}=\sum_{i\not\in I} x_i\leq b.\]
Folglich ist $I$ eine L"osung f"ur PART.
Da die "Uberf"uhrung der Daten in Zeit $O(n)$ m"oglich ist, ist die 
Aussage damit bewiesen.
$\quad\Box$
\vspace*{0.5cm}

Das {\em 3--F"arbungsproblem} ist gegeben durch einen ungerichteten Graphen. 
Frage: K"onnen die Knoten so mit drei Farben eingef"arbt werden, da"s
jeweils zwei durch eine Kante verbundene Knoten unterschiedliche
F"arbungen besitzen? Die \NP--Vollst"andigkeit dieses Problems wurde
bei Stockmeyer [1973] gezeigt.

%Wir betrachten jetzt das Klassifikationsproblem $3 | \cdot |f$. Dies liegt
%nat"urlich auch wieder in \NP, da zu der Frage, ob es einen
%Zielfunktionswert $Z \leq b, \ b \in \N$ gibt, wieder ein Zertifikat $y$
%f"ur jeden "`Ja"'--Input $x$ mit $|x|$ = $O(n)$ existiert, derart da"s 
%$y = \{f(C_1),f(C_2),\dots,f(C_k)\}$
%mit $|y|$ = $O(n)$ und in Zeit $O(n)$ kann dies "uberpr"uft werden.

\begin{satz}
F"ur alle Zielfunktionen $f$ gilt
\[ \mbox{3--F"arbung} \ \alphared \ 3|\cdot|f .\]
\end{satz}
{\bf Beweis:}
In einem ungerichteten Graphen $G=(V,E)$ seien $d_{ij}=1 \ \forall\,
\{i,j\} \in E$. Das hei"st ein 3--Clustering--Problem hat genau dann eine
L"osung mit Zielfunktionswert 0, wenn eine 3--F"arbung existiert.$\quad\Box$
\vspace*{0.5cm}

Unter einer beliebigen Distanzfunktion sind also bereits alle Probleme
mit 3 oder mehr Clustern \NP--schwierig, zudem auch die 2--Cluster--Probleme
$2|\cdot|\max \sum d_{ij}$, $2|\cdot|\sum\sum d_{ij}$ und $
2|\cdot|\sum \frac 1n \sum d_{ij}$ (siehe hierzu Brucker [1978]).


\subsection{Polynomial l"osbare Probleme}
Das Problem $2|\cdot| \max \max d_{ij}$ ist hingegen polynomial l"osbar, 
hierzu hat Brucker [1978] einen Algorithmus angegeben und die Optimalit"at
seiner L"osung gezeigt.
Auch ist bekannt, da"s einige Probleme polynomial l"osbar sind, wenn 
die Distanzen im euklidischen Raum liegen. F"ur das Problem
$2|m|\sum \frac 1n \sum d_{ij}$ hat Bock [1974]
einen $O((\frac nm) 2^{m-1})$--Algorithmus entwickelt; dieser ist
polynomiell f"ur festes $m$.
F"ur $k|1|\sum \frac 1n \sum d_{ij}$ und
$k|1|\max \max d_{ij}$ hat Brucker [1978] ein Verfahren
zur L"osung vorgestellt, das auf folgender Eigenschaft beruht:
%ebenso wie f"ur $k|1|\sum \max d_{ij}$:
%Au"serdem gilt der


\begin{satz}
\label{satzdisjunkt}
F"ur die Probleme $k|1| \sum \max d_{ij}$,
$k|1| \max \max d_{ij}$ 
und das Problem $k|m|\sum \frac 1n \sum d_{ij}$ 
existiert eine optimale L"osung
$C_1^*,\dots,C_k^*$ mit der Eigenschaft, da"s die konvexen H"ullen
der $C_i^*$ disjunkt sind.
\end{satz}
{\bf Beweis: }
Es gen"ugt zu zeigen, da"s jede optimale L"osung in eine mit konvexen
Clustern "uberf"uhrt werden kann, ohne den Zielfunktionswert zu
erh"ohen. O.B.d.A. nehmen wir an, da"s $C_1 = \{x_1,\dots,x_r\}$ 
der Cluster ist, der den kleinsten Wert enth"alt. Sei $y$ ein Punkt
aus der Wertemenge mit $y \not\in C_1$, aber $y$ liegt in der konvexen
H"ulle von $C_1$. Dann k"onnen wir $y$ schreiben als
\[ y \ = \ \sum_{\nu = 1}^r \lambda_\nu x_\nu \quad \mbox{mit }
	\lambda_\nu \geq 0\quad \mbox{und } \sum_{\nu = 1}^r \lambda_\nu =1.\]
F"ur die Abst"ande $d(x_i,y)$ zwischen $x_i\,(i=1,\dots,r)$ und $y$ gilt
\begin{eqnarray*}
 d(x_i,y)   & = & d(x_i,\sum_{\nu=1}^r \lambda_\nu x_\nu)
	        \ \leq \ \sum_{\nu = 1}^r \lambda_\nu d(x_i,x_\nu)\\
			& \leq & \max_{\nu = 1}^r d(x_i,x_\nu)
			\ \leq \ \max_{i,j = 1 \atop i \neq j}^r d(x_i,x_j).
\end{eqnarray*}
Somit wird der Wert der Zielfunkton nicht erh"oht, wenn wir zu $C_1$ alle
Punkte hinzunehmen, die in der konvexen H"ulle von $C_1$ liegen. In 
$C_2,\dots,C_k$ entfernen wir alle diese Punkte und wenden das Verfahren
erneut auf die nichtleeren Cluster $C_2,\dots,C_k$ an und erhalten somit
nach $k$ Schritten eine Partitionierung mit der gew"unschten 
Eigenschaft. F"ur den Fall, da"s die Anzahl der nichtleeren Cluster
kleiner als $k$ geworden ist, m"ussen wir 
einelementige Cluster konstruieren, die die konvexe Eigenschaft der
anderen Cluster nicht verletzen,
%Mengen konstruieren, 
bis wir eine $k$--Partitionierung erhalten.

F"ur den Beweis der Konvexit"at des Problems $k|m|\sum \frac 1n 
\sum d_{ij}$  sei hier auf Bock [1974] verwiesen.
$\quad\Box$
\vspace*{0.5cm}

\begin{satz}
F"ur das Problem $k|1|\sum\max d_{ij}$ existiert eine
optimale L"osung in der die $k-1$ gr"o"sten $d_i$--Werte
($d_i = x_{i+1} - x_i$) die Clustergrenzen darstellen.
\end{satz}
{\bf Beweis:}
Seien $x_1 < x_2 < \cdots < x_n$ die Punkte im 1--dimensionalen
euklidischen $k$--Clustering--Problem, und angenommen, es existiert eine
L"osung derart, da"s die konvexen H"ullen der einzelnen Cluster
voneinander getrennt sind.
Die optimale Partitionierung ist dann gegeben durch Mengen der Form
\begin{eqnarray*}
	C_1^* & = & \{x_1=x_{i_0},\dots,x_{i_1}\},\\
	C_2^* & = & \{x_{i_1 +1},\dots,x_{i_2}\},\\
	      & \vdots & \\
	C_k^* & = & \{x_{i_{k-1}+1},\dots,x_{i_k}=x_n\}.
\end{eqnarray*}
Der entsprechende Zielfunktionswert ist dann
\begin{eqnarray*}
	f(C_1^*,\dots,C_k^*) 
	 	& = & x_{i_1} - x_{i_0} + x_{i_2} - x_{i_1+1}+
		 	\cdots + x_{i_{k-1}} - x_{i_{k-2}+1} + x_{i_k} - x_{i_{k-1}+1}\\
		& = & x_{i_k} - x_{i_0} - \sum_{\nu = 1}^{k-1} x_{i_\nu +1} 
			- x_{i_\nu}\\
		& = & x_n - x_1 - \sum_{\nu = 1}^{k-1} d_{i_\nu}.
\end{eqnarray*}
%Das hei"st also, da"s die $k-1$ gr"o"sten $d_i$--Werte die 
%Clustergrenzen darstellen.--
Die $k-1$ gr"o"sten $d_i$--Werte minimieren also den Funktionswert.
$\quad\Box$
\vspace*{0.5cm}


Die Konvex--Eigenschaft aus Satz \ref{satzdisjunkt} gilt nicht, 
wenn wir die Zielfunktion durch $\max \sum d_{ij}$ oder $\sum \sum d_{ij}$
ersetzen, wie folgendes Beispiel f"ur die Funktion $\sum \sum d_{ij}$
zeigt:

%\begin{beis}
{\bf Beispiel:}
Sei $E=\{0,5,6,7,12\}, \ k=2$. Dann existieren bis auf Symmetrie 2 konvexe
Clusterungen:
\vspace*{3mm}

\input b1-2-1.pic

\input b1-2-2.pic

Betrachten wir jedoch
\vspace*{3mm}

\input b1-2-3.pic

so erhalten wir $f = 12 \ + \ 1 + 1 + 2 = 16.$
%\end{beis}
Zum besseren Verst"andnis der Zahlen wurde bei
der Summierung der Zielfunktionswerte 
jede Strecke nur einmal addiert, da im euklidischen Raum 
$d(x,y) = d(y,x)$ gilt. Die wirklichen Werte sind also doppelt so
gro"s wie hier angegeben, "andern aber nichts an der Aussage.

Die Probleme $k|2|\sum\max d_{ij}$ und $k|2|\max\max d_{ij}$ konnten
Preparata \& Shamos [1985] auf das Auf\/finden der gr"o"sten Distanz
in einem Polygon reduzieren. Dieses Problem zu l"osen ist mit
linearem Aufwand m"oglich.


\section{L"osungsstrategien}
Es ist nicht schwierig einzusehen, da"s es f"ur gr"o"sere,
schwierige Probleme kaum 
m"oglich ist alle Gruppierungen zu berechnen ({\em Methode der totalen 
Enumeration}).
Selbst Verfahren wie die {\em dynamische Programmierung} oder {\em Branch
and Bound} bieten keine wirtschaftliche Alternative, um eine optimale
Gruppierung zu bestimmen.  Dementsprechend
benutzen wir Heuristiken und Approximationsalgorithmen zur L"osung.

Heuristische Verfahren gibt es nahezu beliebig viele (Anderberg [1973]).
Alle haben sie gemeinsam, da"s sie sich an geometrisch--visuellen
Vorstellungen von Cluster--Bildungen von in der Ebene verteilten 
Punktobjekten orientieren. Dies ist sehr vern"unftig, da das visuelle
Klassifizieren von Punkten in der Ebene durch Algorithmen im 
allgemeinen nicht "ubertroffen werden kann.
Tats"achlich sind die Cluster--Algorithmen erst in h"oheren Dimensionen
der in zwei Dimensionen visuell m"oglichen Klassifikation
"uberlegen.

Grunds"atzlich gibt es drei verschiedene Verfahren:
\begin{itemize}
\item {\em hierarchische},
\item {\em graphentheoretische} und
\item {\em numerische}.
\end{itemize}
Erstgenanntes kann man weiter in {\em agglomerative} und {\em divisive}
Verfahren
unterteilen: Bei den agglomerativen geht man von einer Anfangspartition 
aus, in der alle Elemente einen eigenen einzelnen Cluster bilden, 
und man f"ugt nun solange Cluster zusammen, bis die gew"unschte
Clusteranzahl erreicht ist.
Hierf"uhr haben Anderberg [1973] einen $O(n^3)$--Algorithmus und
Day \& Edelsbrunner [1984] einen $O(n^2)$--Algorithmus entwickelt.

Bei den divisiven Methoden verf"ahrt man genau umgekehrt: Von einem
"`Supercluster"' aus startend teilt man diesen in immer kleiner 
werdende Haufen, bis man ebenfalls die gew"unschte Clusteranzahl
erreicht hat. 
Eine typische Vorgehensweise ist hierbei den gr"o"sten verbliebenen
Cluster zweizuteilen. Mit dem hierarchischen Ansatz ist es auch
m"oglich nicht disjunkte, nicht exhaustive Partitionierungen
vorzunehmen.

\begin{figure}[htp]
\[\psfig{figure=b1-8}\]
\caption{Hierarchischer Ansatz}
\end{figure}

Die graphentheoretischen Verfahren basieren auf
Zusammenhangskomponenten, indem man die Elemente miteinander verbindet,
deren Abstand unter einer gewissen Schranke $d$ liegt. Auf diese Weise
versucht man soviele Zusammenhangskomponenten zu konstruieren, wie es
die gew"unschte Clusteranzahl vorgibt.
Eigenschaften dieses Verfahrens sind, da"s keine Zielfunktion ben"otigt
wird und somit auch die entstehenden Cluster nicht notwendig kompakt
("`rund"', "`ellipsoid"' oder "ahnlich) sind, sondern da"s sie sich
kennzeichnenderweise
verzweigen und sehr langgestreckt sein k"onnen.

\begin{figure}[htp]
\[\psfig{figure=b1-9}\]
\caption{Graphentheoretischer Ansatz}
\end{figure}

Die dritte Gruppe bilden die numerischen Verfahren, die in ihrer
Haupterscheinungsform sogenannte Austauschverfahren sind:
Ausgehend von Startl"osungen versuchen sie durch Austauschalgorithmen
eine optimale Partitionierung zu erzielen.
Diese Verfahren sind auch unter dem Namen {\em hill climbing} bekannt.
Eine solche Vorgehensweise beschreitet auch die {\em Lokale Suche}, die
im folgenden Kapitel beschrieben wird.

\begin{figure}[htp]
\begin{center}
\input b3-1.pic
\end{center}
\caption{Darstellung des Austauschverfahrens}
\end{figure}

F"ur den Fall, da"s die Clusteranzahl $k=2$ ist, gibt es noch einige
spezielle L"osungsverfahren, auf die wir hier aber nicht weiter
eingehen wollen, sondern dazu auf Rao [1971] verweisen.





